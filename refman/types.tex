\chapter{Types}\index{type}

Variables are untyped but values are typed. There exist twelve different
types of values: \keyword{null}, Boolean values, integers, strings,
lists, dictionaries, functions, abstract syntax trees, matching
results of a regular expression, control flow graph nodes, input
streams, and output streams.

\section{Null value}\index{null}\label{null}

Null values can have the value \keyword{null} only.
Uninitialized variables are initially bound to \keyword{null}
(see \ref{var}).

\section{Boolean values}\index{Boolean}

Boolean values can be either \ident{true} or \ident{false}.

The standard identifiers \ident{true} and \ident{false}
may be used to denote the two possible Boolean values. In
the following example, a variable named \ident{flag} is
declared and set to \ident{false}. In the following statement
the Boolean value of a logical expression is assigned to \ident{flag}.

\begin{lstlisting}
   var flag = false;
   flag = flag || (i == 7);
\end{lstlisting}

\section{Integer values}\index{integer}\label{integer}

Integer values can be of arbitrary size as their
implementation is based on the GNU Multiple Precision
Library\footnote{\url{https://gmplib.org/}}\index{GNU Multiple Precision
Library}. Please note that floating point\index{floating point} values
are not supported.

\section{String values}\index{string}\label{string}

Strings are sequences of Unicode codepoints which can be of arbitrary length.

The following example demonstrates how implicitly strings can be
converted to integers or integers to strings. Firstly, the variable
\ident{i} is initialized to the integer value 17. Secondly, \ident{j}
is initialized to the concatenation of the strings \lstinline!"17"!
(implicitly converted from the integer value of \textit{i}) and
\lstinline!"8"!. Finally, the sum of \textit{i} and \textit{j} is
computed where \textit{j} is converted to an integer first. These
implicit conversions are enforced by the operators \token{\&}
which expects strings and \token{+} which expects integers.

\begin{lstlisting}
   var i = 17;
   var j = i & "8";
   var k = i + j; // k is set to 195
\end{lstlisting}

When strings are converted to integers and vice versa,
a decimal representation is expected or generated, respectively.
Conversions of strings to integers are implemented using
the \ident{mpz\_set\_str} function of the
GNU Multiple Precision Library\index{GNU Multiple Precision Library}
with base set to 10. This means that initial white space characters are
permitted in the string. However, trailing non-whitespace characters
are not permitted.\footnote{In this regard, the semantics diverts from that of
Perl which evaluates "10x" + 1 as 11.}

\section{Lists}\index{list}\label{lists}

Lists are ordered sets of elements which can be accessed by
using indices $0..n-1$.

The following example shows how lists can be constructed
using the \lstinline![...]!-syntax and how to iterate through
lists using a \keyword{foreach}- or \keyword{while}-statement:

\begin{lstlisting}
   var cities = ["Berlin", "Hamburg", "Stuttgart", "Ulm"];
   foreach city in (cities) {
      println(city);
   }
   // alternatively, indices can be used:
   var index = 0;
   while (index < len(cities)) {
      println(cities[index]);
      ++index;
   }
\end{lstlisting}

\noindent
Expressions of list type reference the data structure that is
actually representing the list. Consequently,
an assignment of a list causes a list to be shared
(shallow copy). Example:

\begin{lstlisting}
   var numbers = [1, 3, 5];
   var more_numbers = numbers;
   push(more_numbers, 7);
   // len(numbers) == 4
\end{lstlisting}

By using the predefined functions \ident{push}\index{push} and
\ident{pop},\index{pop} lists can be used as FIFO queues.\index{queue}

\section{Dictionaries}\index{dictionary}\label{dictionary}

Dictionaries are data structures that map string-valued keys
to values of arbitrary type (including \keyword{null}).

The attribution rule in the following example shows how dictionaries
can be used to construct symbol tables. Abstract syntax trees that are
bound to variables (\ident{id}, \ident{declarator} and \ident{block} in
this example) can be used as dictionaries to attach attributes to a node
of the syntax tree. In this example, \textit{block.vars} designates the
entry of the dictionary \ident{block} selected by the key \ident{vars}.
Whenever a key is constant and an identifier (see \ref{identifier}),
the dot-operator can be used as a selecting operator of dictionaries.
Selector expressions can be put into \lstinline!{...}! as demonstrated
by \lstinline!{id}! in this example. As shown below, dictionaries can be
constructed using the \lstinline!{...}! syntax. In this case the keys in
front of the \token{->} delimiter must be identifiers.

\begin{lstlisting}
   ("direct_declarator" ("identifier" id)) as declarator
         in ("compound_statement" *) as block -> {
      block.vars{id} = {
         decl -> declarator,
         used -> false,
         minlevel -> -1,
         minblock -> null,
      };
   }
\end{lstlisting}

\noindent
Individual keys and their associated entries in a dictionary
may be deleted through a deletion statement.

\bigskip
\noindent
Example:\index{delete}

\begin{lstlisting}
   delete block.vars{id}; // delete entry for id in block.vars
\end{lstlisting}

\noindent
Expressions of type dictionary just reference the data structure
representing the actual dictionary. Consequently, an assignment of a
dictionary causes a dictionary to be shared
(shallow copy).\index{shallow copy}

Dictionaries can also be interpreted as sets\index{sets}\label{sets}
where the key values represent the members and the associated values
are ignored.  Following set operators are supported which expect both
operands to be of the dictionary type:\footnote{The set operators
resemble those of Pascal. Unlike Modula-2 that added an
operator for the symmetric difference, however, \token{/} is
not used for this operator. Instead, \token{\caretSY} has been taken because
C uses it as operator for XOR which is similar to the symmetric
difference.}\index{set operators}

\begin{tabular}{l p{4in}}
   \token{+} & union of two sets \\
   \token{$-$} & difference, i.e. taking the first set
      and removing all keys belonging to the second set \\
   \token{$*$} & intersection of two sets \\
   \token{\caretSY} & symmetric difference, i.e. the result
      includes keys only which belong either to the first or
      to the second set but not to both \\
\end{tabular}

\noindent
All set operators\index{set operators} work on the keys only.
The associated values are taken from the first operand of a
set operator, if present, and otherwise taken from the second.
In addition, the assignment operators \token{+=} and \token{$-$=}
are supported for sets (see \ref{setassignment}).

\section{Functions}\index{function}\label{function}

Function values consist of an optional parameter list, a function body
and a closure\index{closure}.
A closure of an anonymous function\index{function!anonymous} allows a
function body to refer to all lexically visible bindings at the point
where the function was constructed.

In the following example, the anonymous functions assigned to
\ident{incr} and \ident{decr} refer both to the same instance
of the local variable \ident{counter} (see \ref{var}).

\begin{lstlisting}
   var counter = 0;
   var incr = sub { return ++counter; };
   var decr = sub { return --counter; };
\end{lstlisting}

\section{Abstract syntax trees}\index{abstract syntax tree}

\label{ast}
An abstract syntax tree is either a token\index{token} or a
node\index{node}\index{operator node} consisting of an operator
represented by a string and an ordered list of subtrees\index{subtree}.
There exists two kinds of leafs\index{leaf}, i.e. token
leafs or operator nodes with an empty list of subtrees.

Each node (be it a regular node or a leaf node representing a token)
has an associated set of attributes that is organized in the form
of a dictionary.

Tokens provide a textual string in two variants. The token
literal value\index{token!literal value} preserves the original
text sequence while the token text value\index{token!text value}
provides the processed contents of a token. Example: In case of a
string literal \lstinline+"Hello, world!"+, the token literal value
returns the entire sequence including the quotes while the token
text value returns the contents of the string literal without the
quotes. The token literal value and the token text value of a token
can be retrieved through \ident{tokenliteral}\index{tokenliteral} and
\ident{tokentext}\index{tokentext} standard functions, respectively.

Examples: The following function \ident{traverse} traverses an
abstract syntax tree recursively to generate a string representing it.
The standard function \ident{isoperator} is used to distinguish between
operator and token nodes. An operator node can be examined like a list
to retrieve the individual subtrees.

\begin{lstlisting}
   sub traverse (node) {
      var result = "";
      if (isoperator(node)) {
         result = "(" & operator(node);
         foreach operand in (node) {
            result &= " " & traverse(operand);
         }
         result &= ")";
      } else {
         result = tokenliteral(node);
      }
      return result;
   }
\end{lstlisting}

\noindent
Alternatively, the individual operands can be indexed:

\begin{lstlisting}
   sub traverse (node) {
      var result = "";
      if (isoperator(node)) {
         result = "(" & operator(node);
         var index = 0;
         while (index < len(node)) {
            result &= " " & traverse(node[index]);
            ++index;
         }
         result &= ")";
      } else {
         result = tokenliteral(node);
      }
      return result;
   }
\end{lstlisting}

\noindent
The attributes of the root node of an abstract syntax tree
can be examined using the predefined
\textit{extract\_attributes}\index{extract\_attributes}
function (see \ref{predefined}):

\begin{lstlisting}
   foreach (attribute, value) in (extract_attributes(node)) {
      // ...
   }
\end{lstlisting}

\noindent
The following attribution rules matches all \lstinline!"compound_statement"!
operator nodes with an arbitrary number of subnodes. Whenever the rule
is executed, the operator node is bound to \ident{block} due to the
\keyword{as} clause. Abstract syntax trees provide a dictionary that
can be used for per-node attributes. The example below initializes the
attributes \ident{vars}, \ident{level}, and \ident{up} of \ident{block}:

\begin{lstlisting}
   ("compound_statement" *) as block -> {
      block.vars = {};
      block.level = 0;
      block.up = null;
   }
\end{lstlisting}

\noindent
Expressions of type abstract syntax tree reference a node within an
abstract syntax tree. Consequently, an assignment of an abstract syntax
tree causes the tree to be shared.

\section{Matching results of regular expressions}\label{matchresult}

If a regular expression matches, a variable can be bound to its
result. This result consists of the matched text and a list
of captured substrings, if ``\lstinline!(...)!''-constructs
where used within the regular expression. Objects of this
type can be interpreted as strings, in this case the whole
matched text is delivered. Alternatively, when indices are
used, beginning with an index of 0 for the first captured substring,
or when objects of this type are put into a list context,
the captured substrings are delivered.

Examples: In the following transformation rule\index{transformation rule}
(see \ref{trans}),
\ident{dec} is bound to the sequence of digits recognized by the
regular expression. Variables bound to matching results can be
used like scalars to refer to the matched text.

\begin{lstlisting}
   ("decimal_literal" m{^\d+$} as dec) -> ("decimal_literal" {dec + 1})
\end{lstlisting}%stopzone

\noindent
The regular expression of the following example 
captures one substring which is refered to by \lstinline!dec[0]!.

\begin{lstlisting}
   ("decimal_literal" m{^\d+(\d)$} as dec) -> ("decimal_literal" {dec & dec[0]})
\end{lstlisting}%stopzone

\section{Control flow graph nodes}\index{control flow graph}\label{cfg}

Attribution rules can be used to construct a interprocedural control
flow graph in conformance to the IFDS framework by
\cite{Reps:1995}. In this framework, a program is represented by a
set of directed graphs $G = \{G_1, ..., G_n\}$ where each individual
graph $G_i$ represents a procedure. Each graph $G_i = (N_i, E_i, n_i)$
consists of nodes $N_i$, directed edges $E_i$, and a unique name $n_i$.

Nodes are created using the \textit{cfg\_node}\index{cfg\_node} function
which takes one or two parameters. The parameters specify a type and/or
the associated node of an abstract syntax tree:

\begin{lstlisting}
   var node1 = cfg_node(node); // cfg node associated with an ast node
   var node2 = cfg_node("entry"); // cfg node created with the type "entry"
   var cls = cfg_node("close_block", block); // typed cfg node with ast node
\end{lstlisting}

Control flow node types\index{control flow node type} are strings which
can be used as identifiers (see \ref{identifier}). These types can be
queried using the \textit{cfg\_type}\index{cfg\_type} function and used
within conditions of state machines:

\begin{lstlisting}
   /* kill instance when we are leaving our block;
      node is bound to the current cfg node */
   at close_block where node.astnode == block -> close
\end{lstlisting}

Directed edges are created using the function
\textit{cfg\_connect}\index{cfg\_connect} which takes two control flow
graph nodes and optionally a third parameter with a label which is a
string that can be used as an identifier (see \ref{identifier}):

\begin{lstlisting}
   // label "t" represents "true" for conditional paths
   cfg_connect(node1, node2, "t");
\end{lstlisting}

Labels can be used within conditions of state machines.
Following example matches the tree expression against
the abstract syntax node of the current control flow
graph node, requires that the node is typed as
\textit{conditional\_fork} and tests within the
\keyword{if} constructs for the labels ``t'' and ``f'':

\begin{lstlisting}
   ("identifier" varname) as id at conditional_fork
      where exists id.object && id.object == object
      if t and when welldefined -> nonnull
      if t and when isnull -> cut
      if f and when welldefined -> isnull
      if f and when nonnull -> cut
\end{lstlisting}

Either all edges from a node have distinct labels or none
of the edges are labelled.

Each control flow node has (like syntax tree nodes) a set of attributes
that is organized as a dictionary. If a control flow node is associated
to a node of an abstract syntax tree, the \ident{astnode}\index{astnode}
field allows to access it.

The \keyword{foreach} loop may be used to iterate through all successor
nodes of a control flow graph node. In case of labeled edges, the
branches can be examined in the \ident{branch}\index{branch} dictionary.

The set of graphs $G$ is represented by the predefined
dictionary named \ident{graph}\index{graph}. Per convention
this graph shall have the following attributes:

\bigskip
\noindent
\begin{tabular}{l l l}
   \hline
   attribute & type & description \\
   \hline
   \ident{entries} & dictionary & entry nodes by name \\
   \ident{exits} & dictionary & exit nodes by name \\
   \ident{nodes} & list & list of all nodes \\
   \hline
\end{tabular}

\bigskip
\noindent
Expressions of type control flow graph node reference a node within a
control flow graph. Consequently, an assignment of a node causes the
node to be shared.

\section{Streams}\index{stream}
Streams exist in two variants, input and output streams. The
predefined bindings include
\ident{stdin}\index{stdin} (standard input),
\ident{stdout}\index{stdout} (standard output), and
\ident{stderr}\index{stderr} (standard error output)
(see \ref{predefined}).
Files can be opened using the \ident{open}\index{open}
function. By default, \ident{open} returns an input stream
but files can also be opened for writing using \lstinline!"r"!
as second parameter.
The predefined functions \ident{getline}\index{getline},
\ident{println}\index{println}, and \ident{prints}\index{prints}
allow to read or to write to a stream, respectively. Streams
are implicitly closed when the last reference to it is cut.

\begin{lstlisting}
   sub main(argv) {
      foreach arg in (argv) {
         var input = open(arg);
         if (input) {
            var line;
            while (defined(line = getline(input))) {
               println(line);
            }
         } else {
            println(stderr, arg, ": could not be opened for reading");
         }
      }
   }
\end{lstlisting}

\section{Implicit Conversions}\index{conversion}

In dependence of their context, values may be implicitly converted
into another type, if possible. A run time error is raised whenever
a conversion cannot be done.

\subsection{Conversions to string type}\index{string}\label{functionconv}
\label{stringconv}

Null values\index{null} are converted to the empty string,
integers\index{integer} are converted into their decimal representation,
Boolean\index{Boolean} into \lstinline!"1"! or \lstinline!"0"!,
match results\index{match result} into the matched substring, and
flow graph nodes\index{flow graph node} into their type. In case of
abstract syntax trees, a conversion delivers either the result of
the standard function \ident{operator}\index{operator} in case of
operator nodes or \ident{tokentext}\index{tokentext} otherwise. In case
of lists and dictionaries, the size is returned (equivalent to the
\ident{len}\index{len} standard function). Functions without
parameter lists are invoked with
\ident{args} bound to \keyword{null} and their result is converted
into string type. An empty string is delivered if \keyword{null} is
returned.

Streams\index{stream} are converted to their associated file names
which were passed to \ident{open}\index{open}. The predefined
standard streams are converted to the respective names,
i.e. \lstinline!"stdin"!, \lstinline!"stdout"!, and
\lstinline!"stderr"!.

\subsection{Conversions to integer type}\index{integer}\label{intconv}

Null values are converted to 0, Boolean values \ident{false} and
\ident{true} are converted to 0 and 1, respectively. All other types are
converted into a string first and then interpreted as a decimal integer,
i.e. an optional minus sign followed by a non-empty sequence of digits
``0'' to ``9'' of arbitrary length. If the string does not conform to
this format, a run time error is raised.

Streams\index{stream} are converted to 1, if they are in a good
state, and 0 otherwise in case of errors or on encountering end
of input.

\subsection{Conversions to Boolean type}\index{Boolean}\label{boolconv}

Null values are converted to \ident{false}.
Non-zero integer values are converted to \ident{true}, zero to \ident{false}.
All abstract syntax trees, match results, and flow graph nodes are
converted to \ident{true} irrespectively of their value. (Note that
even an empty match result is evaluated as \ident{true} to distinguish
it from a failed match.)

Streams\index{stream} are converted to \ident{true}, if they are
in a good state, and \ident{false} otherwise in case of errors or on
encountering end of input.

Values of other types are converted into
strings first. The string values \lstinline!""! and \lstinline!"0"!
are converted to \ident{false}, all other values to \ident{true}.

\subsection{Conversions to list type}\index{list}

\label{listconv}
A null value is converted into an empty list. An abstract syntax tree
is, if its root is an operator node, converted to a list of subtrees of
its root, or, if it is a token, converted to a list of one string-valued
element representing the token text. A dictionary is converted into a
sorted list of keys. In case of match results, a list of all strings
is generated that were captured by the regular subexpressions. (This
list is empty if there were no subexpressions.) For control flow graph
nodes a list of successor nodes is generated. All other types are
converted into a string and from this string a list is contructed with
that element.

\subsection{Conversions to dictionary type}\index{dictionary}

\label{dictconv}
A null value is converted into an empty dictionary. Syntax
tree nodes are converted to their set of attributes (see \ref{ast}).
All other types are converted to list first (see \ref{listconv}). Then
each element of the list gets converted into a string
(see \ref{stringconv}) which is subsequently used as a key.
For the associated values the Boolean value \ident{true} is used.

\subsection{Conversion to abstract syntax tree type}
\index{abstract syntax tree}

\label{treeconv}
The value is converted to a string and the string value is turned
into an abstract syntax tree consisting just of a token whose
token text is identical to the string. A null value is treated
similarly but an empty string is taken.

\section{Examining the type of an object}\label{exam-type}

The predefined function \ident{type} takes an arbitrary object as
value and returns its type as string:

\bigskip
\noindent
\begin{tabular}{l l}
   \hline
   object type & returned string \\
   \hline
   \keyword{null} & \lstinline!"null"! \\
   Boolean value & \lstinline!"boolean"! \\
   integer value & \lstinline!"integer"! \\
   string value & \lstinline!"string"! \\
   list & \lstinline!"list"! \\
   dictionary & \lstinline!"dictionary"! \\
   function & \lstinline!"function"! \\
   abstract syntax tree & \lstinline!"tree"! \\
   matching results & \lstinline!"match_result"! \\
   control flow graph node & \lstinline!"flow_graph_node"! \\
   input stream & \lstinline!"istream"! \\
   output stream & \lstinline!"ostream"! \\
   \hline
\end{tabular}

\endinput
