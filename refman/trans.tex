\chapter{Transformations}\label{trans}

Transformation rules allow to generate new abstract syntax trees
on base of a given tree, or to modify an abstract syntax tree in-place.
Like attribution rules, transformation rules can be collected in named
rule sets which are executed only by invoking them.

\section{Regular transformation rules}\label{regtrans}

Regular transformation rules do not belong to a named rule set and do
not operate in-place. Instead, if specified, they allow to generate
mutants in conjunction with the printing rules. They are executed in the
course of a regular execution order implicitly at the end if print rules
exist (see \ref{xorder}). If a \ident{main}\index{main} function is
defined, they are processed as soon as the execution of the \ident{main}
function is finished. Regular transformation rules are not executed
in case of a free-standing script (see \ref{free-xorder}).

In dependence of the command line arguments, a set of files is generated
where each file contains a variant of the original source where exactly
one transformation rule at one point in the abstract syntax tree
was executed. The print rules are free to use the special bindings
\ident{location}\index{location} and \ident{rulename}\index{location}
to insert the information of the transformed location and the rulename,
if given, in the generated output (see \ref{predefined}).

If the set of regular transformation rules is executed and none of
them matches, a runtime exception is raised (``no matching transformation
rule found'').

\begin{grammar}
   \nonterminal{transformation-rules}
      \produces \lexkeyword{transformation} \lexkeyword{rules}
	 \lextoken{\{} \nonterminal{transformations}
	 \lextoken{\}} \\
   \nonterminal{transformations}
      \produces \nonterminal{transformation} \\
      \produces \nonterminal{transformations}
	 \nonterminal{transformation} \\
   \nonterminal{transformation}
      \produces \optional{\nonterminal{identifier} \lextoken{:}}
	 \nonterminal{conditional-tree-expression}
	 \lextoken{->} \nextline
	 \nonterminal{transformation-instructions} \\
      \produces \optional{\nonterminal{identifier} \lextoken{:}}
	 \nonterminal{conditional-tree-expression}
	 \lextoken{->} \nextline \lexkeyword{pre}
	 \nonterminal{transformation-instructions} \\
      \produces \optional{\nonterminal{identifier} \lextoken{:}}
	 \nonterminal{conditional-tree-expression}
	 \lextoken{->} \nextline \lexkeyword{post}
	 \nonterminal{transformation-instructions} \\
\end{grammar}

\noindent
The right-hand side of a transformation rule consists of the replacement
tree construct and optional blocks that are executed before and after
the actual transformation takes place.

\begin{grammar}
   \nonterminal{transformation-instructions}
      \produces \optional{\nonterminal{pre-transformation-block}} \nextline
	 \nonterminal{subnode-constructor} \nextline
	 \optional{\nonterminal{post-transformation-block}} \\
   \nonterminal{pre-transformation-block}
      \produces \nonterminal{block} \\
   \nonterminal{post-transformation-block}
      \produces \nonterminal{block} \\
\end{grammar}

\noindent
Example: Following simple mutating transformation rule replaces
an addition by a multiplication and exchanges the two operands:

\begin{lstlisting}
transformation rules {
   ("+" op1 op2) -> ("*" op2 op1)
}
\end{lstlisting}

\section{Named sets of generating transformation rules}\label{named-trrules}

Named sets of generating transformation rules are not implicitly
executed unlike the regular transformation rules (see \ref{xorder}).
Instead the given name is bound to a function that

\begin{itemize}
   \item expects an abstract syntax tree
      as parameter or, if no parameter is given, uses \ident{root},
   \item generates a clone for each instance where a
      transformation rule matches a node of the abstract syntax tree,
   \item applies the matching transformation rule on the matching node, and
   \item returns a list of the transformed clones.
\end{itemize}

\noindent
If no transformation rule matches, an empty list is returned.

Within this rule set, \ident{root}\index{root} is bound to the
to be transformed abstract syntax tree.

\begin{grammar}
   \nonterminal{named-transformation-rules}
      \produces \lexkeyword{transformation} \lexkeyword{rules}
	 \nonterminal{identifier}
	 \lextoken{\{} \nonterminal{transformations}
	 \lextoken{\}}
\end{grammar}

\section{Named sets of in-place transformation rules}
\label{named-inplace-trrules}

Named sets of in-place transformation rules are not implicitly executed
like the regular transformation rules (see \ref{xorder}). Instead the
given name is bound to a function that

\begin{itemize}
   \item expects a to be transformed abstract syntax tree
      as parameter or, if no parameter is given, uses \ident{root},
   \item performs all the transformations in-place,
   \item suppresses the execution of any possibly conflicting rule, and
   \item returns the number of transformations performed.
\end{itemize}

Please note that in-place transformation rules never match the
given root node to preserve the referential integrity.
Within this rule set, \ident{root}\index{root} is bound to the
to be transformed abstract syntax tree.

\begin{grammar}
   \nonterminal{named-inplace-transformation-rules}
      \produces \lexkeyword{inplace}
	 \lexkeyword{transformation} \lexkeyword{rules}
	 \nonterminal{identifier} \nextline
	 \lextoken{\{} \nonterminal{transformations}
	 \lextoken{\}}
\end{grammar}

Due to the suppression of possible conflicts, in-place
transformation functions are usually embedded in a loop
that apply the transformations repeatedly until none
of the transformation rules fire:

\begin{lstlisting}
inplace transformation rules simplification_rules {
   ("->" op name) -> ("." ("pointer_dereference" op) name)
   ("[]" array index) -> ("pointer_dereference" ("+" array index))
   /* ... */
}

sub simplify(tree) {
   while (simplification_rules() > 0) {}
}
\end{lstlisting}

\endinput
