\chapter{State machines}

\section{Regular state machines}
\index{state machine!regular}

A state machine consists of a finite number of named states, a set of
variables, and a set of rules. Rules can create instances of a state machine,
change their states, execute arbitrary code, or stop their execution.
State machines without creation rules are called \textit{global}
state machines\index{state machine!global}. All other state machines are
called \textit{local}\index{state machine!local}.\footnote{This property
conforms to the chosen abstract state machine (see below). However,
it is well possible to have a local state machine which is derived from
\textit{global} or \textit{global\_tracker}. In this case, you will get
an interprocedural analysis for a local object.}

State machines may be derived from an abstract state machine. In this
case, the name of the abstract state machine is to be given after a
colon.

\begin{grammar}
   \nonterminal{state-machine}
      \produces \lexkeyword{state} \lexkeyword{machine} \nonterminal{identifier}
	 \lextoken{(}
	 \nonterminal{states} \lextoken{)} \nextline
	 \lextoken{\{} \nonterminal{sm-vars} \nonterminal{sm-functions}
	    \nonterminal{sm-rules}
	 \lextoken{\}} \\
      \produces \lexkeyword{state} \lexkeyword{machine}
	 \nonterminal{identifier} \lextoken{:} \nonterminal{identifier}
	 \lextoken{(} \nonterminal{states} \lextoken{)} \nextline
	 \lextoken{\{} \nonterminal{sm-vars} \nonterminal{sm-functions}
	    \nonterminal{sm-rules}
	 \lextoken{\}}
\end{grammar}%stopzone

\noindent
State machine instances are executed during a recursive traverse
of the control flow graph. Initially, all global state machines
are instantiated. Local state machines are instantiated whenever a
\keyword{create} clause fires the first time for a control flow graph
node. State machines instances continue to run through the traverse
until they are explicitly stopped. If a node of the control flow graph
has multiple exits, a copy of a state machine is created for each of
the possible paths.

All global state machine instances derived from the same state machine are
\textit{related}\index{state machines!related} to each other. All local
state machine instances that are derived from the same instance that was
created at a particular control flow graph node are also \textit{related}
to each other.

Variables declared within a state machine are
either \textit{shared}\index{variable!shared} or
\textit{private}\index{variable!private}. Shared variables are shared
among all state machine instances that are related to each other.
Private variables are never shared among instances of a state machine
but kept private for each path taken by a state machine.

\begin{grammar}
   \nonterminal{sm-vars}
      \produces \optional{\nonterminal{sm-var-declarations}} \\
   \nonterminal{sm-var-declarations}
      \produces \nonterminal{sm-var-declaration} \\
      \produces \nonterminal{sm-var-declarations}
	 \nonterminal{sm-var-declaration} \\
   \nonterminal{sm-var-declaration}
      \produces \lexkeyword{shared} \lexkeyword{var} \nonterminal{identifier} \\
      \produces \lexkeyword{shared} \lexkeyword{var}
	 \lextoken{=} \nonterminal{expression} \\
      \produces \lexkeyword{private} \lexkeyword{var}
	 \nonterminal{identifier} \\
      \produces \lexkeyword{private} \lexkeyword{var}
	 \nonterminal{identifier} \lextoken{=} \nonterminal{expression}
\end{grammar}

\noindent
Local functions can be declared within a state machine. They inherit
the scope from their state machine instance including all shared and
private variables.

\begin{grammar}
   \nonterminal{sm-functions}
      \produces \optional{\nonterminal{sm-function-definitions}} \\
   \nonterminal{sm-function-definitions}
      \produces \nonterminal{function-definition} \\
      \produces \nonterminal{sm-function-definitions}
	 \nonterminal{function-definition}
\end{grammar}

\noindent
State machines are executed if a control flow graph\index{control
flow graph} has been constructed through the attribution rules and if
\ident{graph.root}\index{graph.root} exists and points to a node of the
control flow graph.

State machines instances, when executed, have a state and begin initially
with the very first state of their list of named states. State machines
traverse through the control flow graph. The execution of a state
machine instance stops if a control flow graph node has already been
visited by a related state machine instance with the same state. For
every edge leaving the actual control flow graph node, a new instance of a
state machine is constructed which inherits the state and the values of
the private variables from its predecessor and continues to share the
shared variables with its predecessors. All these new instances are
related to each other and to the instance they have been derived from.

While traversing a control flow graph, the rules of a state machine
can consider

\begin{enumerate}
   \item[(1)] the node of the abstract syntax tree associated with the
      control flow graph node including its associated data structures,
      if existant (see \nonterminal{conditional-tree-expression}),
   \item[(2)] the current node of the control flow graph includes its
      type and its associated data structures
      (using the \keyword{at} keyword,
      see \nonterminal{cfg-node-expression}),
   \item[(3)] the label of the edge leaving the current node of the control
      flow graph (using the \keyword{if} keyword,
      see \nonterminal{cfg-edge-condition}), and
   \item[(4)] the current state (using the \keyword{when} keyword,
      see \nonterminal{sm-state-condition}).
\end{enumerate}

\noindent
Combinations of (3) and (4) can be grouped (see
\nonterminal{sm-alternatives}) and associated to a combination of (1)
and (2) (see \nonterminal{sm-condition}):

\begin{grammar}
   \nonterminal{sm-rule}
      \produces \nonterminal{sm-condition} \lextoken{->}
	 \nonterminal{sm-block} \optional{\lextoken{;}} \\
      \produces \nonterminal{sm-condition} \nonterminal{sm-alternatives} \\
      \produces \nonterminal{sm-condition} \lextoken{->}
	 \lexkeyword{create} \nonterminal{block} \optional{\lextoken{;}} \\
      \produces \lexkeyword{on} \lexkeyword{close}
	 \nonterminal{sm-handler} \optional{\lextoken{;}} \\
   \nonterminal{sm-condition}
      \produces \nonterminal{conditional-tree-expression} \\
      \produces \lexkeyword{at} \nonterminal{cfg-node-expression} \\
      \produces \nonterminal{conditional-tree-expression}
	 \lexkeyword{at} \nonterminal{cfg-node-expression} \\
   \nonterminal{cfg-node-expression}
      \produces \nonterminal{cfg-node-types} \\
      \produces \nonterminal{cfg-node-types}
	 \lexkeyword{where} \nonterminal{expression} \\
      \produces \lextoken{*} \\
      \produces \lextoken{*} \lexkeyword{where} \nonterminal{expression} \\
   \nonterminal{cfg-node-types}
      \produces \nonterminal{identifier} \\
      \produces \nonterminal{cfg-node-types} \lexkeyword{or}
	 \nonterminal{identifier} \\
   \nonterminal{sm-alternatives}
      \produces \nonterminal{sm-alternative} \optional{\lextoken{;}} \\
      \produces \nonterminal{sm-alternatives}
	 \nonterminal{sm-alternative} \optional{\lextoken{;}} \\
   \nonterminal{sm-alternative}
      \produces \lexkeyword{if} \nonterminal{cfg-edge-condition}
	 \lextoken{->} \nonterminal{sm-block} \\
      \produces \lexkeyword{when}
	 \nonterminal{sm-state-condition} \lextoken{->}
	 \nonterminal{sm-block} \\
      \produces \lexkeyword{if} \nonterminal{cfg-edge-condition}
	 \lexkeyword{and} \nextline \lexkeyword{when}
	 \nonterminal{sm-state-condition} \lextoken{->}
	 \nonterminal{sm-block} \\
   \nonterminal{cfg-edge-condition}
      \produces \nonterminal{identifier} \\
      \produces \nonterminal{cfg-edge-condition} \lexkeyword{or}
	 \nonterminal{identifier} \\
   \nonterminal{sm-state-condition}
      \produces \nonterminal{identifier} \\
      \produces \nonterminal{sm-state-condition} \lexkeyword{or}
	 \nonterminal{identifier}
\end{grammar}

\noindent
If a star is given in a \nonterminal{cfg-node-expression}, all
control flow graph nodes are matched. This allows to write a
rule that fires in all cases:

\begin{lstlisting}
   at * -> {
      // ...
   }
\end{lstlisting}

\noindent
The expressions that are given as conditions to
\nonterminal{conditional-tree-expression} or
\nonterminal{cfg-node-expression} have access to all shared and
private variables of the state machine. In addition, the name
\ident{node} is bound to the control flow graph node in the
condition of \nonterminal{cfg-node-expression} (but not in the
condition of \nonterminal{conditional-tree-expression}). (The node
of the abstract syntax tree, if defined, can be accessed through
\ident{node.astnode}.\index{node.astnode})

When a rule fires, i.e. if its condition is true, an optional action
gets performed and an optional block is executed. If multiple rule
conditions are true, the associated actions and blocks are executed
in the order of appearance.

\begin{grammar}
   \nonterminal{sm-block}
      \produces \nonterminal{block} \\
      \produces \nonterminal{sm-action} \\
      \produces \nonterminal{sm-action} \nonterminal{block} \\
   \nonterminal{sm-action}
      \produces \lexkeyword{cache}
	 \lextoken{(} \nonterminal{expression} \lextoken{)} \\
      \produces \lexkeyword{close} \\
      \produces \lexkeyword{cut} \\
      \produces \lexkeyword{retract} \\
      \produces \nonterminal{identifier}
\end{grammar}

\noindent
The actions \keyword{close}, \keyword{cut}, and \keyword{retract} stop
the execution of a state machine instance. In addition, \keyword{close}
causes the close handler to be invoked, and \keyword{retract} undoes the
visit of that particular control flow graph node. Whenever a state machine
instance stops, the stop is immediate, i.e. no further rules are
evaluated.

The \keyword{cache} action expects a parameter designating a control
flow graph node and checks if the successor node has already been
visited. If this has been the case, the state machine continues at the
given node with the resulting states of the state machine instance
with the same entry state. This allows an interprocedural execution of
state machines. For this to work, a function call requires at least two
special nodes, one for the call and one for the return position. The
function call is linked to the entry node of the function which must
be used by all function calls to that function. The exit node of the
function is expected to be linked to all return nodes that are paired
with a call node to that function. Note that the state machine must
explicitly cut off all the return nodes that are not paired with the
corresponding call node (see the example below). The close handler,
if defined, is invoked if the instance that hit the function first is
unable to find a path leading to the exit node. As \keyword{cache}
considers the actual state when deciding how to go on, the current
state must not be changed by any of the rules matching the particular
control flow graph node.

If a state machine stops at the end of a path, i.e. if the
action \lexkeyword{close} is executed or if the current node of
the control flow graph has no edges departing from it, all close
handlers are invoked, possibly in dependence of the current state.

\begin{grammar}
   \nonterminal{sm-handler}
      \produces \lextoken{->} \nonterminal{block} \\
      \produces \lexkeyword{when} \nonterminal{sm-state-condition}
	 \lextoken{->} \nonterminal{block}
\end{grammar}

\noindent
Within a \nonterminal{cfg-node-expression}, the blocks, and close
handlers all local variables and functions of the state machine
including those inherited from the abstract state machines are visible.
In addition, the variables bound by a
\nonterminal{conditional-tree-expression} and the following list
of bindings is visible within a \nonterminal{cfg-node-expression}
or a block:

\bigskip
\noindent
\begin{tabularx}{\textwidth}{l l X}
   \hline
   name & type & description \\
   \hline
   \ident{current\_state} & string &
      textual representation of the current state when
      the current control flow graph node was entered \\
   \ident{label} & string &
      string representation of the label of the current edge
      in the control flow graph; this string is empty if no
      such label has been defined \\
   \ident{node} & control flow graph node &
      refers to the actual control flow graph node;
      note that \lstinline!node.astnode!, if it exists,
      refers to the associated node of the abstract syntax tree \\
   \hline
\end{tabularx}

\bigskip
\noindent
Example: The following state machine is derived from \textit{global\_tracker},
an abstract state machine which is itself
derived from \ident{global} (see \ref{smglobal}) and which supports
the \ident{err} method that delivers a backtrace documenting the
path that lead to an error. This state machine tracks all global
interprocedural paths and checks that the invocations of \ident{lock}
and \ident{unlock} are properly balanced:

\begin{lstlisting}
state machine lock_checker: global_tracker (unlocked, locked, broken) {
   ("function_call" ("identifier" "lock")) at before_call
      when unlocked -> locked
      when locked -> broken { err("lock possibly called twice"); }
   ("function_call" ("identifier" "unlock")) at before_call
      when locked -> unlocked
      when unlocked -> broken { err("unlock possibly called unbalanced"); }

   on close when locked -> {
      if (!endless_recursion) {
         err("missing invocation of unlock at end of path");
      }
   }
   on close -> {
      if (endless_recursion) {
         err("endless recursion detected");
      }
   }
}
\end{lstlisting}

\section{Abstract state machines}
\index{state machine!abstract}

Abstract state machines allow to factorize common rule sets out of
regular state machines. Abstract state machines are never instantiated.
If a state machine (be it regular or abstract) is derived from an
abstract state machine, all its variables and rules are inherited.

All blocks associated to a rule of a state machine share the same
scope. This includes all blocks inherited from the abstract state
machine a regular state machine has been derived from. In consequence,
all variables of a state machine, independent from the location
of their declaration, are visible to all blocks of that state machine.

\begin{grammar}
   \nonterminal{abstract-state-machine}
      \produces \lexkeyword{abstract} \lexkeyword{state}
	 \lexkeyword{machine} \nonterminal{identifier} \nextline
	 \lextoken{\{} \nonterminal{sm-vars} \nonterminal{sm-functions}
	    \nonterminal{sm-rules}
	 \lextoken{\}} \\
      \produces \lexkeyword{abstract} \lexkeyword{state}
	 \lexkeyword{machine} \nonterminal{identifier}
	 \lextoken{:} \nonterminal{identifier} \nextline
	 \lextoken{\{} \nonterminal{sm-vars} \nonterminal{sm-functions}
	    \nonterminal{sm-rules}
	 \lextoken{\}}
\end{grammar}

\noindent
Example: The follow abstract state machine serves as common rule set
of global state machines. Global state machines do not follow
the local links at a function call but follow the extern links to
the entry node of the called function. Whenever they return from
the exit node of a called function back to the rtn node paired
to the call node they cut all paths that are not properly nested.
The nesting is checked through the private variable \ident{chain}.

\index{state machine!global}
\label{smglobal}
\begin{lstlisting}
abstract state machine global {
   private var chain = {nestlevel -> 0};
   at call
      if local -> cut;
      if extern -> cache(node.branch.local) {
	 chain = {
	    nestlevel -> chain.nestlevel + 1,
	    next -> chain,
	    callid -> node.callid
	 };
      }
   at rtn where chain.nestlevel == 0 || chain.callid != node.callid -> retract;
   at rtn where chain.nestlevel > 0 && chain.callid == node.callid -> {
      chain = chain.next;
   }
}
\end{lstlisting}

\endinput
