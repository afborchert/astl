\chapter{Units}

A compilation unit begins with library and import clauses and
provides an arbitrary number of rules and definitions:

\begin{grammar}
   \nonterminal{unit}
      \produces \optional{\nonterminal{clauses}}
	 \optional{\nonterminal{rules}} \\
   \nonterminal{clauses}
      \produces \nonterminal{clause} \\
      \produces \nonterminal{clauses} \nonterminal{clause} \\
   \nonterminal{clause}
      \produces \nonterminal{import-clause} \\
      \produces \nonterminal{library-clauses} \\
      \produces \nonterminal{operator-set-clause} \\
   \nonterminal{import-clause}
      \produces \lexkeyword{import} \nonterminal{identifier} \lextoken{;} \\
   \nonterminal{library-clause}
      \produces \lexkeyword{library} \nonterminal{string-literal}
	 \lextoken{;} \\
   \nonterminal{rules}
      \produces \nonterminal{rule} \\
      \produces \nonterminal{rules} \nonterminal{rule} \\
   \nonterminal{rule}
      \produces \nonterminal{function-definition} \\
      \produces \nonterminal{attribution-rules} \\
      \produces \nonterminal{state-machine} \\
      \produces \nonterminal{abstract-state-machine} \\
      \produces \nonterminal{transformation-rules} \\
      \produces \nonterminal{named-transformation-rules} \\
      \produces \nonterminal{named-inplace-transformation-rules} \\
      \produces \nonterminal{operator-rules} \\
      \produces \nonterminal{print-rules} \\
      \produces \nonterminal{named-print-rules}
\end{grammar}

\section{Libraries}

Libraries are source files ending in the suffix ``.ast'' that can
be loaded through an import clause and are looked for in all
directories of the library path. The library path consists initially
of the current directory only. Library clauses add the given directories
to the end of the library path. Libraries must conform to the
same syntax, i.e. they are considered as \nonterminal{unit}. All their
rules and definitions are added to the global pool of rules and definitions.
Multiple attempts to import the same library unit are permitted. In this
case, just the first import clause is executed.

Multiple operator rules (see \ref{oprules}) are not permitted.

\section{Operator set clauses}\label{opset}

Operator sets can be defined through operator set clauses:

\begin{grammar}
   \nonterminal{operator-set-clause}
      \produces \lexkeyword{opset} \nonterminal{identifier}
	 \lextoken{=} \nonterminal{operator-expression} \lextoken{;}
\end{grammar}

\noindent
Operator expressions (see \ref{opexpr}) can be used within
tree expressions (see \ref{treeexpr}) and operator rules
(see \ref{oprules}). Operator set clauses must textually precede the
tree expressions or operator rules using them.

\section{Order of appearance}\label{order}

In case of attribution, transformation, and print rules the order
of appearance is significant. As all libraries can contribute to the
global rule sets, their order is defined as follows:

\begin{enumerate}
   \item The first source file (usually presented at the command line)
      is considered first.
   \item Within a source file, the order is preserved.
   \item Import directives are executed after a source file has been
      loaded.
   \item For each imported unit which has not been loaded yet, the
      same process starts recursively.
\end{enumerate}

\section{Regular rule sets}
Regular rule sets, i.e. regular attribution rules (see \ref{reg-attr}),
regular transformation rules (see \ref{regtrans}), and
regular print rules (see \ref{regprint}), can be spread over
multiple units and are implicitly united in the order of appearance
(see \ref{order}). This allows, for example, the main program which
is loaded first to override a particular print rule by defining
a print rule within its unit.

\section{Global scope}

The global scope includes all predefined bindings (see \ref{predefined}),
global functions (see \ref{functiondef}), named attribution rules
(see \ref{named-atrules}), named generating transformation rules (see
\ref{named-trrules}), named in-place transformation rules
(see \ref{named-inplace-trrules}), and named print rules
(see \ref{named-print}). In case of the function definitions,
the order of appearance does not matter as all global functions see all
other global functions.  The name of state machines do not belong to
the global bindings, i.e. a global function name can conflict with the
name of a state machine but this is not recommended.

\endinput
