\chapter{Attribution rules}\label{attr}

\section{Regular attribution rules}\label{reg-attr}
An attribution rule consists of a tree expression and a block.
When attribution rules are executed, the abstract syntax tree
is traversed depth-first. For each visited node of the abstract
syntax tree, all attribution rules are executed whose tree
expressions match the visited node. When multiple rules match
for the same node they are executed in the order of
appearance in the source.

Attributes rules are executed in pre- or postorder in respect
to the traversal of the subnodes. By default, rules are executed
in preorder. The keywords \keyword{pre} and \keyword{post} can
be used to specify the order.

\begin{grammar}
   \nonterminal{attribution-rules}
      \produces \lexkeyword{attribution} \lexkeyword{rules}
	 \lextoken{\{} \nonterminal{attributions}
	 \lextoken{\}} \\
   \nonterminal{attributions}
      \produces \nonterminal{attribution} \\
      \produces \nonterminal{attributions} \nonterminal{attribution} \\
   \nonterminal{attribution}
      \produces \nonterminal{conditional-tree-expressions}
	 \lextoken{->} \nonterminal{block} \\
      \produces \nonterminal{conditional-tree-expressions}
	 \lextoken{->} \lexkeyword{pre} \nonterminal{block} \\
      \produces \nonterminal{conditional-tree-expressions}
	 \lextoken{->} \lexkeyword{post} \nonterminal{block} \\
\end{grammar}

\noindent
Example: In the following example we have three attribution rules
that focus on operator nodes with the operator
\lstinline!"compound_statement"!. The first rule is executed
first. Afterwards, the second rule is executed whenever a
compound statement is nested within another compound statement.
Then all the attribution rules are executed that match the subnodes.
Finally, the third rule is executed last as it is a postfix rule.

\begin{lstlisting}
   ("compound_statement" *) as block -> {
      block.vars = {};
      block.level = 0;
      block.up = null;
   }
   ("compound_statement" *) as inner_block
         in ("compound_statement" *) as outer_block -> {
      inner_block.level = outer_block.level + 1;
      inner_block.up = outer_block;
   }
   ("compound_statement" *) as block in ("translation_unit" *) as root -> post {
      foreach varname in (block.vars) {
	 var entry = block.vars{varname};
	 if (entry.used) {
	    if (entry.minblock != block) {
	       println(location(entry.decl) & ": variable " & varname &
		  " should be moved into " & location(entry.minblock));
	    }
	 } else {
	    println(location(entry.decl) & ": unused variable " & varname);
	 }
      }
   }
\end{lstlisting}

\section{Named attribution rules}\label{named-atrules}

Named attribution rules are similar to regular attribution rules but
they are not implicitly executed (see \ref{xorder}). The given name
is bound to a function which executes the associated attribute rule
set. The function takes one optional parameter that specifies the
root node of the to be traversed tree. Otherwise, if no parameter is
given, \ident{root} is taken (see \ref{predefined}).
Within this rule set, \ident{root}\index{root} is bound to the
to be attributed abstract syntax tree.

\begin{grammar}
   \nonterminal{attribution-rules}
      \produces \lexkeyword{attribution} \lexkeyword{rules}
	 \nonterminal{identifier}
	 \lextoken{\{} \nonterminal{attributions}
	 \lextoken{\}}
\end{grammar}

\endinput
