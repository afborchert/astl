\chapter{Function definitions}\index{function!definition}\label{functiondef}

\section{Regular global functions}

Functions can be defined at the global level. They are
globally visible and their closures\index{closure} are restricted
to the set of predefined bindings and the other global functions.

\begin{grammar}
   \nonterminal{function-definition}
      \produces \keyword{sub} \nonterminal{identifier} \nonterminal{block} \\
      \produces \keyword{sub} \nonterminal{identifier}
	 \nonterminal{parameter-list} \nonterminal{block}
\end{grammar}

\noindent
Parameter passing is handled as for locally constructed functions
(see \ref{funcon} and \ref{call}):

\begin{itemize}
   \item If a parameter list\index{parameter list} is given,
      the number of actual arguments must match the length of the
      parameter list and each of the named parameters is bound to the
      corresponding actual argument on invocation.
   \item Otherwise, if no parameter list is given, any number
      of actual parameters is permitted, and passed through
      the list \ident{args}.\index{args}
\end{itemize}

\noindent
Example: The following function computes the greatest common divisor
of the two arguments \ident{a} and \ident{b}:

\begin{lstlisting}
sub gcd (a, b) {
   while (a != b) {
      if (a > b) {
	 a -= b;
      } else {
	 b -= a;
      }
   }
   return a;
}
\end{lstlisting}

\noindent
Each function returns a value. If the function ends without
executing \keyword{return} or if the \keyword{return}-statement
is without value, \keyword{null} is returned. Otherwise, the
value of the \keyword{return} expression is returned (see \ref{return}).

Local functions\index{function!local} can be defined using function
constructors (see \ref{funcon}) and within state machines (see
\ref{sm-localfun}).

\section{Main function}\index{main}\index{function!main}

The global \ident{main} function, if it exists, is invoked implicitly
within the regular execution order (see \ref{xorder}) or is controlling
the entire execution in case of a free-standing script
(see \ref{free-xorder}). The remaining
command line\index{command line} arguments are passed as list
to \ident{main}, either through an explicitly named parameter
or, if the parameter list has been omitted, through the variable
\ident{args}\index{args} which is locally bound within \ident{main}
(see \ref{call}):

\begin{lstlisting}
/* variant with explicit parameter list */
sub main (argv) {
   foreach arg in (argv) {
      /* process arg... */
   }
}

/* variant without explicit parameter list */
sub main {
   foreach arg in (args) {
      /* process arg... */
   }
}
\end{lstlisting}

\noindent
The argument list passed to \ident{main} includes neither the
binary of the Astl interpreter nor the name of the script or
any other options that are consumed by the runtime environment. The
name of the script can be obtained through the predefined binding
\ident{cmdname}.

The return value of the \ident{main} function is ignored.

\section{Other global functions}\index{function!global}

Named attribution rules (see \ref{named-atrules}),
named generating transformation rules (see \ref{named-trrules}),
named in-place transformation rules (see \ref{named-inplace-trrules}),
and named print rules (see \ref{named-print}) can be
used like regular global functions. They all expect an
abstract syntax tree as argument (see \ref{ast}).

\endinput
