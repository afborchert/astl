\chapter{Function definitions}\index{function!definition}\label{functiondef}

Global functions can be declared at the global level. They are
globally visible and their closures\index{closure} are restricted
to the set of predefined bindings and the other function definitions.

\begin{grammar}
   \nonterminal{function-definition}
      \produces \keyword{sub} \nonterminal{identifier} \nonterminal{block} \\
      \produces \keyword{sub} \nonterminal{identifier}
	 \nonterminal{parameter-list} \nonterminal{block}
\end{grammar}

\noindent
Parameter passing is handled as for locally constructed functions
(see \ref{funcon} and \ref{call}):

\begin{itemize}
   \item If a parameter list is given, the number of actual
      arguments must match the length of the parameter list
      and each of the named parameters is bound to the corresponding
      actual argument on invocation.
   \item Otherwise, if no parameter list is given, any number
      of actual parameters is permitted, and passed through
      the list \ident{args}.
\end{itemize}

Example: The following function computes the greatest common divisor
of the two arguments \ident{a} and \ident{b}:

\begin{lstlisting}
sub gcd (a, b) {
   while (a != b) {
      if (a > b) {
	 a -= b;
      } else {
	 b -= a;
      }
   }
   return a;
}
\end{lstlisting}

Each function returns a value. If the function ends without
executing \keyword{return} or if the \keyword{return}-statement
is without value, \keyword{null} is returned. Otherwise, the
value of the \keyword{return} expression is returned (see \ref{return}).

\noindent
Local functions\index{functions!local} can be defined using function constructors
(see \ref{funcon}) and within state machines (see \ref{sm-localfun}).

Named attribution rules (see \ref{named-atrules}),
named generating transformation rules (see \ref{named-trrules}),
named in-place transformation rules (see \ref{named-inplace-trrules}),
and named print rules (see \ref{named-print}) can be
used like global functions.

\endinput
