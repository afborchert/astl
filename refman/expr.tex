\chapter{Expressions}\index{expression}

Expressions serve to compute values. They can have side effects.
Following table summarizes all operators along with their priorities and
associativity:\index{operator table}\index{priority}\index{associativity}

\setlength\LTleft{0pt}
\setlength\LTright{0pt}
\begin{longtable}{>{\raggedright\hspace{0pt}}p{1in}
      @{\extracolsep{\fill}} p{1.5in} l r l}
   \hline
   Operators & Function & Type & Priority & Associativity \\
   \hline
   \endhead
   \hline \multicolumn{5}{r}{\emph{Continued on the next page}}
   \endfoot
   \hline
   \endlastfoot
   identifiers and literals & simple symbols & primary & 14 & --- \\
   \lstinline!d.f! & field selection & postfix & 14 & left to right \\
   \lstinline!d{f}! & field selection & postfix & 14 & left to right \\
   \lstinline!a[i]! & element selection & postfix & 14 & left to right \\
   \lstinline!exists d.f! & existance operator & prefix & 13 & left to right \\
   \lstinline!++i! & prefix increment & prefix & 13 & non-associative \\
   \lstinline!i++! & postfix increment & postfix & 13 & non-associative \\
   \lstinline!--i! & prefix decrement & prefix & 13 & non-associative \\
   \lstinline!i--! & postfix decrement & postfix & 13 & non-associative \\
   \lstinline!(e)! & grouping & prefix & 13 & non-associative \\
   \lstinline![]! & list aggregate & prefix & 13 & non-associative \\
   \lstinline!{}! & dictionary aggregate & prefix & 13 & non-associative \\
   \lstinline!<()>! & tree construction & prefix & 13 & non-associative \\
   \lstinline!f()! & function call & postfix & 13 & left to right \\
   \lstinline!sub {}! & function constructor & prefix & 13 & non-associative \\
   \lstinline!-! & negation & prefix & 12 & non-associative \\
   \lstinline+!+ & logical negation & prefix & 12 & non-associative \\
   \lstinline!^! & power & infix & 11 & right to left \\
   \lstinline!* div mod!
      & multiplicative operators & infix & 10 & left to right \\
   \lstinline!+ -! & additive operators & infix & 9 & left to right \\
   \lstinline!< > <= >= == !=! & comparison operators & infix & 8 & left to right \\
   \lstinline!x! & repetition & infix & 7 & left to right \\
   \lstinline!&! & concatenation & infix & 6 & left to right \\
   \lstinline!=~! & regexp match & infix & 5 & non-associative \\
   \lstinline!&&! & logical and & infix & 4 & left ro right\\
   \lstinline!||! & logical or & infix & 3 & left to right \\
   \lstinline!? :! & selection & infix & 2 & non-associative \\
   \lstinline!= &= += -=!
      & assignment & prefix & 1 & right to left \\
\end{longtable}

\section{Designators}\index{designator}

A designator allows to specify an object that can be assigned to.
Possible designators are named variables or selected elements of
a variable:

\begin{grammar}
   \nonterminal{designator}
      \produces \nonterminal{identifier} \\
      \produces \nonterminal{designator}
         \lextoken{.} \nonterminal{identifier} \\
      \produces \nonterminal{designator}
         \lextoken{\{} \nonterminal{expression} \lextoken{\}} \\
      \produces \nonterminal{designator} \lextoken{\leftbracketSY}
         \nonterminal{expression} \lextoken{\rightbracketSY} \\
\end{grammar}

\noindent
Identifiers refer to one of the bound or locally declared variables.
The visibility of a variable name is determined by its lexical scope.
In case of name conflicts, the innermost declaration takes precedence.

Objects of dictionary\index{dictionary} or abstract syntax
tree\index{abstract syntax tree} type accept a selector in the form
of a key\index{key} using the ``.''-construct for identifiers or the
\lstinline!{...}!-construct using arbitrary expressions which are
converted into a string. Unused keys may only be used if no further
selectors are following and if the designator is used at the left side
of an assignment or as parameter to the \keyword{exists} operator.
Otherwise, the use of a non-existing key causes a run-time error to be
raised.

Objects of list\index{list}, match result\index{match result},
or abstract syntax tree\index{abstract syntax tree} type
accept a selector in the form of a list index using the
\lstinline![...]!-construct using arbitrary expressions which are
converted into an integer (see \ref{intconv}).
(In case of abstract syntax trees,
an index is only accepted if the root element of the tree is
an operator node.) Indices must fall into the range $[0..n-1]$
where $n$ is the length of the list or match result or the number
of operands in case of an abstract syntax tree. Individual
elements of a match result that represent captured substrings or
subtrees selected out of an abstract syntax tree must
not be assigned to. Lists can be extended only through the
\ident{push}\index{push} standard function.

\section{Increment and decrement operators}
\index{increment operators}\index{decrement operators}

Designators can be incremented or decremented using prefix
or postfix operators:

\begin{grammar}
   \nonterminal{prefix-increment}
      \produces \lextoken{++} \nonterminal{designator} \\
   \nonterminal{prefix-decrement}
      \produces \lextoken{$-${}$-$} \nonterminal{designator} \\
   \nonterminal{postfix-increment}
      \produces \nonterminal{designator} \lextoken{++} \\
   \nonterminal{postfix-decrement}
      \produces \nonterminal{designator} \lextoken{$-${}$-$}
\end{grammar}

\noindent
In case of the prefix increment and decrement operators,
the resulting value of the expression is the value of
the designator \emph{after} it has been incremented or decremented.
In case of the postfix increment and decrement operators,
the resulting value of the expression is the value
the designator \emph{before} it has been incremented or decremented.

\section{List aggregates}\index{list!aggregate}\index{aggregate!list}

List aggregates construct list values:

\begin{grammar}
   \nonterminal{list-aggregate}
      \produces \lextoken{\leftbracketSY} \lextoken{\rightbracketSY} \\
      \produces \lextoken{\leftbracketSY}
         \nonterminal{expression-list} \lextoken{\rightbracketSY} \\
   \nonterminal{expression-list}
      \produces \nonterminal{expression} \\
      \produces \nonterminal{expression-list} \lextoken{,}
         \nonterminal{expression}
\end{grammar}

\noindent
Nested list aggregates construct nested data structures. Example:

\begin{lstlisting}
   var matrix = [[1 2 3], [4 5 6], [7 8 9]];
\end{lstlisting}

\section{Dictionary aggregates}
\index{dictionary!aggregate}\index{aggregate!dictionary}

Dictionary aggregates construct dictionary values:

\begin{grammar}
   \nonterminal{dictionary-aggregate}
      \produces \lextoken{\{} \lextoken{\}} \\
      \produces \lextoken{\{} \nonterminal{key-value-pairs} \lextoken{\}} \\
      \produces \lextoken{\{} \nonterminal{key-value-pairs}
         \lextoken{,} \lextoken{\}} \\
   \nonterminal{key-value-pairs}
      \produces \nonterminal{key-value-pair} \\
      \produces \nonterminal{key-value-pairs} \lextoken{,}
         \nonterminal{key-value-pair} \\
   \nonterminal{key-value-pair}
      \produces \nonterminal{identifier} \lextoken{->}
         \nonterminal{expression} \\
      \produces \nonterminal{string-literal} \lextoken{->}
         \nonterminal{expression}
\end{grammar}

\noindent
Identifiers on the left-hand-side of a \token{->} operator do not
refer to equally-named variables in the current lexical scope but are
interpreted literally. In the following example, \ident{member} is a
dictionary with one entry where the key is ``name'' and its associated
value ``Andreas'':

\begin{lstlisting}
   var name = "Andreas";
   var member = {name -> name};
\end{lstlisting}

\noindent
Key values which are not of string type are converted implicitly
to strings (see \ref{stringconv}). Null values are not permitted
as keys.

\section{Abstract syntax tree constructors}\label{treecon}
\index{abstract syntax tree!constructor}\index{syntax tree!constructor}

Tree constructors create new abstract syntax trees:

\begin{grammar}
   \nonterminal{tree-constructor}
      \produces \lextoken{<(} \nonterminal{string-literal} \lextoken{)>} \\
      \produces \lextoken{<(} \nonterminal{string-literal}
	 \nonterminal{subnodes-constructor} \lextoken{)>} \\
   \nonterminal{subnode-constructor}
      \produces \nonterminal{identifier} \\
      \produces \nonterminal{identifier} \lextoken{...} \\
      \produces \nonterminal{tree-expression-constructor} \\
      \produces \nonterminal{tree-expression-constructor}
	 \lexkeyword{as} \nonterminal{identifier} \\
      \produces \lextoken{\{} \nonterminal{expression} \lextoken{\}} \\
      \produces \lextoken{\{} \nonterminal{expression} \lextoken{\}}
	 \lexkeyword{as} \nonterminal{identifier} \\
      \produces \lextoken{\{} \nonterminal{expression} \lextoken{\}}
	 \lextoken{...} \\
      \produces \lextoken{\{} \nonterminal{expression} \lextoken{\}}
	 \lextoken{...} \lexkeyword{as} \nonterminal{identifier} \\
   \nonterminal{tree-expression-constructor}
      \produces \lextoken{(} \nonterminal{string-literal}
	 \lextoken{)} \\
      \produces \lextoken{(} \nonterminal{string-literal}
	 \nonterminal{subnodes-constructor} \lextoken{)} \\
   \nonterminal{subnodes-constructor}
      \produces \nonterminal{subnode-constructor} \\
      \produces \nonterminal{subnodes-constructor}
	 \nonterminal{subnode-constructor} \\
\end{grammar}

If an expression in curly braces or a bound variable name is followed
by the ``...'' operator, the expression or variable must be a list of
subnodes (see \ref{lists}) whose elements are inserted at the
corresponding place.

The syntax is close to tree expressions (see \ref{treeexpr}) except
that the topmost parentheses require angle brackets and that all
constructs are omitted that are required for matching a tree expression.
Alternatively, the predefined functions \ident{make\_node}
and \ident{make\_token} (see \ref{makenode}) can be used.

The following example constructs abstract syntax trees that correspond to
$3 + 4 * 5$ and $2 * (3 + 4 * 5)$, respectively:

\begin{lstlisting}
   var node = <("+"
      { make_token("3") }
      ("*"
	 { make_token("4") }
	 { make_token("5") }
      )
   )>;
   var expr = <("*" { make_token("2") } node)>;
\end{lstlisting}

\noindent
Alternatively, \ident{make\_node} can be used:

\begin{lstlisting}
   var node = make_node("+",
      make_token("3"),
      make_node("*",
	 make_token("4"),
	 make_token("5")
      )
   );
   var expr = make_node("*", make_token("2"), node);
\end{lstlisting}

\section{Function constructors}\index{function!constructor}\label{funcon}

Function constructors define local functions.\index{function!local}
Their lexical closure,\index{lexical closure}
i.e. the lexical scope at the location of the function construction
including the instances of the local variables visible at the time
when the function was constructed, remains visible within
the body of the function and continues to exist as long as the function
can be refered to.

\begin{grammar}
   \nonterminal{function-constructor}
      \produces \lexkeyword{sub} \nonterminal{block} \\
      \produces \lexkeyword{sub} \nonterminal{parameter-list}
	 \nonterminal{block} \\
   \nonterminal{parameter-list}
      \produces \lextoken{(} \lextoken{)} \\
      \produces \lextoken{(} \nonterminal{identifier-list}
	 \lextoken{)} \\
   \nonterminal{identifier-list}
      \produces \nonterminal{identifier} \\
      \produces \nonterminal{identifier-list} \lextoken{,}
	 \nonterminal{identifier}
\end{grammar}

If no parameter list is specified, the number of parameters
is variable. All actual arguments are then converted into
a list and bound to \ident{args}. Otherwise, if a parameter
list is given, the number of parameters is fixed and the
formal parameters are bound to the corresponding actual
parameters.

\section{Function calls}\index{function!invocation}\label{call}

Functions without parameter lists can be invoked with an
arbitrary number of parameters.
All parameters are evaluated left to right and their results
bound to the elements of the list \ident{args}. This list is
empty if no parameters are passed. Otherwise, if the function
has a parameter list, the number of actual arguments must
match the number of named parameters.

\begin{grammar}
   \nonterminal{function-call}
      \produces \nonterminal{primary} \lextoken{(} \lextoken{)} \\
      \produces \nonterminal{primary}
         \lextoken{(} \nonterminal{expression-list} \lextoken{)}
\end{grammar}

\noindent
The value of a function call is determined by its return value
(see \ref{return}). If a function has no \keyword{return}
statement, \keyword{null} is returned.

Note that functions without parameter lists can also be implicitly
invoked through an implicit conversion of a function value to a string
(see \ref{functionconv}). In this case, \ident{args} is bound to
\keyword{null}.

\section{Primaries}

\begin{grammar}
   \nonterminal{primary}
      \produces \nonterminal{designator} \\
      \produces \lexkeyword{exists} \nonterminal{designator} \\
      \produces \lexkeyword{exists}
         \lextoken{(} \nonterminal{designator} \lextoken{)} \\
      \produces \nonterminal{prefix-increment} \\
      \produces \nonterminal{prefix-decrement} \\
      \produces \nonterminal{postfix-increment} \\
      \produces \nonterminal{postfix-decrement} \\
      \produces \lextoken{(} \nonterminal{assignment} \lextoken{)} \\
      \produces \nonterminal{list-aggregate} \\
      \produces \nonterminal{dictionary-aggregate} \\
      \produces \nonterminal{tree-constructor} \\
      \produces \nonterminal{function-call} \\
      \produces \nonterminal{function-constructor} \\
      \produces \nonterminal{cardinal-literal} \\
      \produces \nonterminal{string-literal} \\
      \produces \lexkeyword{null}
\end{grammar}

\noindent
The \keyword{exists} operator returns true if a given key of a designator
is used within a dictionary or not. Example:

\begin{lstlisting}
var dict = {a -> "a", b -> null};
var b1 = exists dict.a; // is true
var b2 = exists dict.b; // is true, even if the value is null
var b3 = defined(dict.b); // is false as dict.b is null
var b4 = exists dict.c; // is false
var b5 = exists dict.c.d; // not permitted as dict.c does not exist
\end{lstlisting}

\section{Factors}

\begin{grammar}
   \nonterminal{factor}
      \produces \nonterminal{primary} \\
      \produces \lextoken{$-$} \nonterminal{primary} \\
      \produces \lextoken{!} \nonterminal{primary}
\end{grammar}

\noindent
The unary minus operator \token{$-$} converts its operand into an integer
and toggles its sign. The logical negation operator \token{!} converts
its operand into a boolean value and negates it.

\section{Power expressions}

\begin{grammar}
   \nonterminal{power-expression}
      \produces \nonterminal{factor} \\
      \produces \nonterminal{power-expression}
         \lextoken{\caretSY} \nonterminal{factor}
\end{grammar}

\noindent
If both operands are dictionaries, the binary operator \token{\caretSY}
is interpreted as set operator implementing the symmetric difference
(see \ref{sets}).\index{set operators!symmetric difference}

Otherwise, the binary operator \token{\caretSY} converts its operands
$a$ and $b$ to integer values and returns $a^b$. Note that
the value of $b$ must be non-negative and not larger than $2^{31}-1$.

\section{Multiplicative expressions}

\begin{grammar}
   \nonterminal{multiplicative-expression}
      \produces \nonterminal{power-expression} \\
      \produces \nonterminal{multiplicative-expression}
         \lextoken{$*$} \nonterminal{power-expression} \\
      \produces \nonterminal{multiplicative-expression}
         \lexkeyword{div} \nonterminal{power-expression} \\
      \produces \nonterminal{multiplicative-expression}
         \lexkeyword{mod} \nonterminal{power-expression}
\end{grammar}

\noindent
If both operands are dictionaries, the binary operator \token{$*$}
is interpreted as intersection\index{set operators!intersection}
operator for sets (see \ref{sets}).

Otherwise, all multiplicative operators convert their operands into integer
values (see \ref{intconv}) and return an integer value.

The operators \keyword{div} and \keyword{mod} follow
Knuth's definition of these operators if $b \not= 0$ \cite{Knuth:divmod}:

\begin{eqnarray*}
   a ~ \mbox{\keyword{div}} ~ b & := &
      \left\lfloor \frac{a}{b} \right\rfloor \\
   a ~ \mbox{\keyword{mod}} ~ b & := &
      a - b \left\lfloor \frac{a}{b} \right\rfloor
\end{eqnarray*}

\noindent
A runtime error is raised if the second operand is zero.

\section{Additive expressions}

\begin{grammar}
   \nonterminal{additive-expression}
      \produces \nonterminal{multiplicative-expression} \\
      \produces \nonterminal{additive-expression}
         \lextoken{$+$} \nonterminal{multiplicative-expression} \\
      \produces \nonterminal{additive-expression}
         \lextoken{$-$} \nonterminal{multiplicative-expression}
\end{grammar}

\noindent
If both operands are dictionaries, the additive operators
are interpreted as set operators (see \ref{sets}),
\token{+} delivering the union\index{set operators!union},
and \token{$-$} the difference\index{set operators!difference}.

Otherwise, all additive operators convert their operands into integer
values (see \ref{intconv}) and return an integer value.

\section{Comparison operators}

\begin{grammar}
   \nonterminal{comparison}
      \produces \nonterminal{additive-expression} \\
      \produces \nonterminal{comparison}
         \lextoken{==} \nonterminal{additive-expression} \\
      \produces \nonterminal{comparison}
         \lextoken{!=} \nonterminal{additive-expression} \\
      \produces \nonterminal{comparison}
         \lextoken{<} \nonterminal{additive-expression} \\
      \produces \nonterminal{comparison}
         \lextoken{<=} \nonterminal{additive-expression} \\
      \produces \nonterminal{comparison}
         \lextoken{>=} \nonterminal{additive-expression} \\
      \produces \nonterminal{comparison}
         \lextoken{>} \nonterminal{additive-expression}
\end{grammar}

\noindent
In case of the comparison operators \token{==} (equality)
and \token{!=} (inequality), \keyword{null} is considered equal
to \keyword{null} only. Non-scalar types, i.e. lists, dictionaries,
functions, abstract syntax trees, matching results of regular
expressions, and control flow graph nodes, are considered equal
only if they refer to the same object.

If one of the operands of integer type (see \ref{integer}),
the other operand, if it is not of integer type, is converted to an
integer value (see \ref{intconv}). The comparison is then performed
on two integers.

Otherwise, both operands are converted to string type (see \ref{stringconv})
and the comparison is then performed octet by octet in comparing
their numerical values in the local encoding.

\section{String repetition}

\begin{grammar}
   \nonterminal{repetitive-expression}
      \produces \nonterminal{comparison} \\
      \produces \nonterminal{repetitive-expression}
         \lexkeyword{x} \nonterminal{comparison}
\end{grammar}

\noindent
The repetition operator \token{x} converts its left operand into a string
(see \ref{stringconv}) and its right operand into an integer
(see \ref{intconv}). If the right operand is zero or negative,
the empty string is returned. Otherwise, a string is generated
which consists of $n$ repetitions of the left operand where
$n$ is the value of the right operand. Example:

\begin{lstlisting}
var fred2 = "fred" x 2; /* result is "fredfred" */
var spaces = " " x 80; /* 80 space characters */
\end{lstlisting}

\section{String and list concatenation}

\begin{grammar}
   \nonterminal{concatenation-expression}
      \produces \nonterminal{repetitive-expression} \\
      \produces \nonterminal{concatenation-expression}
         \lextoken{\&} \nonterminal{repetitive-expression}
\end{grammar}

\noindent
If any of the two operands is of list type, the other operand is, if
necessary, converted into a list (see \ref{listconv}) and a
concatenated list is returned consisting of all elements of the
first operand, followed by those of the second operand.

Otherwise, both operands are converted to strings (see \ref{stringconv})
and the concatenated string of the first and second operand is returned.

\section{Matching strings against regular expressions}

\begin{grammar}
   \nonterminal{match-expression}
      \produces \nonterminal{concatenation-expression} \\
      \produces \nonterminal{concatenation-expression}
         \lextoken{=\tildeSY} \nonterminal{regular-expression} \\
      \produces \nonterminal{concatenation-expression}
         \lextoken{=\tildeSY} \nonterminal{concatenation-expression}
\end{grammar}

\noindent
The matching operator converts its left operand into a string
(see \ref{stringconv}) and matches its against the regular
expression\index{regular expression}. If any substring of the
left operand matches the regular expression, an object of match
result\index{match result} type (see \ref{matchresult})
is constructed and returned. Otherwise, \keyword{null} is returned.

The regular expression can be either specified by a literal
(see \ref{regexlit}) or by an arbitrary expression. In the latter
case, it is converted to a string (see \ref{stringconv}) and
then interpreted as an regular expression like the corresponding
literals but without the enclosing braces.

The regular expression engine is aware of the encoding
in UTF-8\index{UTF-8}:
\begin{itemize}
   \item A dot in the regular expression does not match a byte
      but a Unicode codepoint.
   \item Captures select sequences of Unicode codepoints.
   \item Unicode extensions for regular expressions are
      supported.\footnote{See
      \url{http://www.pcre.org/current/doc/html/pcre2syntax.html}
      for details.}
\end{itemize}

Invalid regular expressions cause runtime exceptions.

Example: The following function \textit{ast\_summary} takes
an abstract syntax tree node as argument and returns a
text representation of it. If this representation extends over
multiple lines, the intermediate lines are replaced by dots:

\begin{lstlisting}
sub ast_summary (tree) {
   var text = gentext(tree);
   var res;
   if (res = text =~ m{([^\n]*)\n.*\n(.*)}) {
      text = res[0] & " ... " & res[1];
   } elsif (res = text =~ m{([^\n]*)\n}) {
      text = res[0] & " ...";
   }
   return text;
}
\end{lstlisting}

\section{Logical expressions}

\begin{grammar}
   \nonterminal{and-expression}
      \produces \nonterminal{match-expression} \\
      \produces \nonterminal{and-expression}
         \lextoken{\&\&} \nonterminal{match-expression} \\
   \nonterminal{or-expression}
      \produces \nonterminal{and-expression} \\
      \produces \nonterminal{or-expression}
         \lextoken{\barSY\barSY} \nonterminal{and-expression}
\end{grammar}

\noindent
The logical operators evaluate the first operand and convert
it to Boolean type (see \ref{boolconv}). If in case of the
logical and operator \token{\&\&} the first operand evaluates
to \ident{false}, the logical expression returns \ident{false}
without evaluating the right operand (\textit{short circuit
evaluation}\index{short circuit evaluation}). Similarly, if
in case of the logical or operator \token{\barSY\barSY} the
first operand evaluates to \ident{true}, the logical expression
returns \ident{true} without evaluating the right operand.

Otherwise, the second operand is evaluated and converted to
Boolean type and its value is returned.

\section{Conditional expressions}

\begin{grammar}
   \nonterminal{conditional}
      \produces \nonterminal{and-expression} \\
      \produces \nonterminal{and-expression} \lextoken{?}
         \nonterminal{and-expression} \lextoken{:}
         \nonterminal{and-expression}
\end{grammar}

\noindent
The conditional operator \token{?} evaluates its first operand
and converts it to Boolean type (see \ref{boolconv}). If it
is \ident{true}, the second operand is evaluated and returned,
otherwise the third.

\section{Assignments and expressions}\label{assignments}

\begin{grammar}
   \nonterminal{assignment}
      \produces \nonterminal{conditional} \\
      \produces \nonterminal{designator} \lextoken{=}
         \nonterminal{assignment} \\
      \produces \nonterminal{designator} \lextoken{\&=}
         \nonterminal{assignment} \\
      \produces \nonterminal{designator} \lextoken{+=}
         \nonterminal{assignment} \\
      \produces \nonterminal{designator} \lextoken{$-$=}
         \nonterminal{assignment} \\
   \nonterminal{expression}
      \produces \nonterminal{assignment}
\end{grammar}

\noindent
The regular assignment operator \token{=} implements reference semantics,
i.e. values do not get copied or cloned but references are shared. In
case of elementary data types like strings or integers, the effect is
not seen as values do not change. Instead, even in case of an increment
or decrement operator, new values are represented by new objects. The
effect, however, can be observed in case of updatable structured data types
like lists and dictionaries. Example:

\begin{lstlisting}
   var l1 = [1, 2, 3];
   var l2 = l1; // both, l1 and l2, refer to the same list
   l1[0] = 4;
   // l2[0] == 4;
\end{lstlisting}

\noindent
The updating assignment operators \token{\&=}, \token{+=}, and \token{-=},
if used on elementary data types (i.e. strings or integers),
compute a new value from the old value of the left-hand-side
and the given right-hand-side and assign it to the designator.
In case of lists and the \token{\&=} operator, the referenced list
is extended by the right-hand-side.

\label{setassignment}
The assignment operators \token{+=} and \token{-=} are interpreted
as corresponding set operators if the designator is a
dictionary.\index{set operators!assignment operators}
In this case, if the right operand is converted into a dictionary,
if necessary (see \ref{dictconv}). Example:

\begin{lstlisting}
   var set = {};
   set += 1; set += 2;
   // set: {1 -> true, 2 -> true}
   set -= 1;
   // set: {2 -> true}
\end{lstlisting}

\noindent
Note that \lstinline!set1 += set2! is a short form of
\lstinline!set1 = set1 + set2!, i.e. a reference to a newly
created dictionary is assigned to. The old dictionary remains
untouched. This behaviour is different from following loop
which extends the existing dictionary:

\begin{lstlisting}
   foreach (key, value) in (set2) {
      if (!exists set1{key}) {
	 set1{key} = value;
      }
   }
\end{lstlisting}

\endinput
