\chapter{Execution}
The order of execution depends on the actual implementation. But
there exist some variants which are supported by the Astl library.
Custom implementations can divert from this.

\section{Standard execution order}\label{xorder}
This is the most common order of execution where
execution starts with the construction of the abstract syntax tree
which is not controlled by the Astl program but by its environment.
In case of syntax errors, the execution is aborted.

Following steps are performed if a valid and non-empty
abstract syntax tree is present:

\begin{enumerate}
   \item Regular attribution rules (see \ref{reg-attr}) are executed,
      if they exist.
   \item State machines (see \ref{sm}) are executed, if they exist
      and a control flow graph has been constructed using attribution rules.
   \item The \textit{main}\index{main} function is executed, if it exists.
   \item The regular transformation rules are executed, provided
      that regular print rules exist (see \ref{regprint}).
      For each transformation, a new abstract syntax tree is generated
      and printed, in dependence of the command line arguments, either
      to standard output or into individual output files.
\end{enumerate}

\section{Free-standing execution order}\label{free-xorder}
Alternatively, some implementations support a so-called free-standing execution
order\index{execution order!free-standing}
where no abstract syntax tree is generated at the
beginning. Instead, the entire execution is controlled by the
\ident{main}\index{main} function.

The main function can process its command line arguments,
open file arguments, and parse them using the \ident{parse}\index{parse}
function (which does not belong to the standard set of predefined functions).
Parsing, if successful, delivers an abstract syntax tree which is
immediately processed by the regular attribution rules. Otherwise,
a string is returned with an error message.

The execution
of state machines can be initiated through
\ident{run\_state\_machines}\index{run\_state\_machines}. This is not
done automatically to give the opportunity to construct a tree from
multiple input sources:

\begin{lstlisting}
sub main(argv) {
   var trees = [];
   foreach arg in (argv) {
      var input = open(arg);
      var res = parse(input);
      if (type(res) == "string") {
	 println(stderr, "parse of ", input, " failed:\n", res);
      } else {
	 push(trees, res);
      }
   }
   var super_root = <("super-root" { trees }...)>;
   run_state_machines(super_root);
}
\end{lstlisting}

Regular transformation rules (see \ref{regtrans})
will never fire in case of free-standing scripts.

\endinput
