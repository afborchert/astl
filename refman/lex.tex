\chapter{Lexical elements}

\section{Source files}

Astl program source files\index{source file} must end in the
suffix ``.ast''\index{suffix} which is to be omitted when
source files are refered to.

\section{Character set}\index{character set}

Astl program sources are to be encoded in UTF-8\index{UTF-8}. While
most Astl tokens including identifiers consist of ASCII\index{ASCII}
characters only, arbitrary non-ASCII characters are permitted in
string literals (see \ref{stringlit}), program text literals (see
\ref{textlit}), regular expression literals (see \ref{regexlit}), and
comments.

Case is significant.

\section{Lines and columns}\label{lines}

To provide locations\index{location} for error messages,
lines\index{line} and columns\index{column} are tracked while reading a
program source\index{source file}. The newline\index{newline} character
(ASCII 10) is interpreted as line terminator\index{line terminator}
and considered otherwise as being equivalent to the space\index{space}
character (ASCII 32). Tabs\index{tab} (ASCII 9) are similarly treated as
space characters but cause the current column to be advanced to the next
multiply of 8 plus 1. Unicode\index{Unicode} codepoints are counted
as one character independently from the number of bytes required to encode
them in UTF-8\index{UTF-8}.

\section{Comments}\index{comment}

Comments can be delimited by ``/*'' and ``*/''. Nested comments
are not supported. Alternatively, comments start with ``//'' and
are ended by the next line terminator\index{line terminator}.
Comments are handled like space characters.

\section{Tokens}\index{token}

Each program source is converted into a sequence of tokens during
the lexical analysis\index{lexical analysis}. Tokens can be
delimiters\index{delimiter} consisting of one or more special
characters\index{special character}, reserved keywords\index{keyword},
identifiers\index{identifier}, or literals\index{literal}. Tokens end if
the next character is a space or if the next character cannot be added
to it.

Example: The character sequence \lstinline!a[i2]+=exists f{i2}? 12: existsf!
consists of the following tokens:

\begin{tabular}{l l}
   \lstinline!a! & an identifier \\
   \lstinline![! & a single-character delimiter \\
   \lstinline!i2! & an identifier \\
   \lstinline!]! & a single-character delimiter \\
   \lstinline!+=! & a compound delimiter \\
   \lstinline!exists! & a keyword \\
   \lstinline!f! & an identifier \\
   \lstinline!{! & a single-character delimiter \\
   \lstinline!i2! & an identifier \\
   \lstinline!}! & a single-character delimiter \\
   \lstinline!?! & a single-character delimiter \\
   \lstinline!12! & a decimal literal \\
   \lstinline!:! & a single-character delimiter \\
   \lstinline!existsf! & an identifier \\
\end{tabular}

\section{Single-character delimiters}\index{delimiter}

Following delimiters consist of one character only and cannot
be part of a longer token which is not a literal:

\bigskip
\noindent
\begin{tabular}{l l l l}
   \lstinline!{! & \lstinline!}! & \lstinline![! & \lstinline!]! \\
   \lstinline!^! & \lstinline!*! & \lstinline!.! & \lstinline!,! \\
   \lstinline!?! & \lstinline!:! & \lstinline!;! \\
\end{tabular}

\section{Compound delimiters}\index{delimiter}

Following delimiters consist of one or multiple characters:

\bigskip
\noindent
\begin{tabular}{l l l l}
   \lstinline!(! & \lstinline!)! & \lstinline!<(! & \lstinline!)>! \\
   \lstinline!<! & \lstinline!<=! & \lstinline!>! & \lstinline!>=! \\
   \lstinline!-! & \lstinline!->! & \lstinline!--! & \lstinline!-=! \\
   \lstinline!+! & \lstinline!+=! & \lstinline!++! \\
   \lstinline!&! & \lstinline!&&! & \lstinline!&=! \\
   \lstinline/!/ & \lstinline/!=/ \\
   \lstinline!=! & \lstinline!=~! & \lstinline!==! \\
   \lstinline!||! \\
\end{tabular}

\section{Keywords}\index{keyword}

Following keywords are reserved. They cannot be used as identifiers.

\bigskip
\noindent
\begin{tabular}{l l l l l}
\keyword{abstract} & \keyword{div} & \keyword{library} & \keyword{pre} & \keyword{transformation} \\
\keyword{and} & \keyword{else} & \keyword{machine} & \keyword{print} & \keyword{var} \\
\keyword{as} & \keyword{elsif} & \keyword{mod} & \keyword{private} & \keyword{when} \\
\keyword{at} & \keyword{exists} & \keyword{nonassoc} & \keyword{retract} & \keyword{where} \\
\keyword{attribution} & \keyword{foreach} & \keyword{null} & \keyword{return} & \keyword{while} \\
\keyword{cache} & \keyword{if} & \keyword{on} & \keyword{right} & \keyword{x} \\
\keyword{close} & \keyword{import} & \keyword{operators} & \keyword{rules} \\
\keyword{create} & \keyword{in} & \keyword{opset} & \keyword{shared} \\
\keyword{cut} & \keyword{inplace} & \keyword{or} & \keyword{state} \\
\keyword{delete} & \keyword{left} & \keyword{post} & \keyword{sub} \\
\end{tabular}


\section{Identifiers}\index{identifier}\label{identifier}

Identifiers begin with a letter, i.e. ``A'' to ``Z'' and ``a'' to ``z'',
or an underscore ``\_'', and are optionally continued with a sequence
of more letters, underscores, or decimal digits ``0'' to ``9''.
Some identifiers are predefined (see \ref{predefined}).

\section{Literals}\index{literal}

Literals can be decimal literals, string literals (see \ref{stringlit}),
regular expressions (see \ref{regexlit}, and program text literals (see \ref{textlit}).

\subsection{Decimal literals}\index{decimal literal}\index{literal!decimal}

Decimal literals begin with a decimal digit\index{digit} ``0'' to ``9'' and
optionally more digits. Decimal constants are unsigned and can
be of arbitrary size.

\subsection{String literals}\index{string literal}\index{literal!string}\label{stringlit}

String literals are delimited by ``"''. Backslashes\index{backslash},
i.e. ``\lstinline!\!'',
are escape characters, i.e. they remove the special meaning of
the following character or allow to insert special characters into a
string:

\bigskip

\noindent
\begin{tabular}{| l l l |}
\hline
new-line & NL (LF) & \lstinline!\n! \\
horizontal tab & HT & \lstinline!\t! \\
vertical tab & VT & \lstinline!\v! \\
backspace & BS & \lstinline!\b! \\
form feed & FF & \lstinline!\f! \\
alert & BEL & \lstinline!\a! \\
space & & \lstinline!\ ! \\
backslash & \backslashSY{} & \lstinline!\\! \\
question mark & ? & \lstinline!\?! \\
single quote & ' & \lstinline!\'! \\
double quote & " & \lstinline!\"! \\
left brace & \{ & \lstinline!\{! \\
right brace & \} & \lstinline!\}! \\
dollar & \$ & \lstinline!\$! \\
octal number & \textit{ooo} & \lstinline!\!\textit{ooo} \\
16-bit Unicode codepoint in hex & \textit{hhhh} & \lstinline!\u!\textit{hhhh} \\
32-bit Unicode codepoint in hex & \textit{hhhhhhhh}
   & \lstinline!\U!\textit{hhhhhhhh} \\
\hline
\end{tabular}

\bigskip

\noindent
(This includes the set of \nonterminal{simple-escape-sequence} and
\nonterminal{octal-escape-sequence} of the C language as defined
by the ISO standard 9899-1999.)

Examples: \lstinline!"Hello, world"! represents the text ``Hello, world'',
\lstinline!"\""! represents the double quote ``"'',
\lstinline!"\\"! represents the backslash ``\lstinline!\!'',
and \lstinline!"Two\nlines"! represents ``Two'', followed by
a line terminator, and ``lines''. Unicode codepoints can be
specified using four (using \lstinline!\u!) or height hex digits
(using \lstinline!\U!). Example: \lstinline^"\U0001f37a Cheers!"^.

\subsection{Regular expressions}\label{regexlit}
\index{regular expression}\index{literal!regular expression}

A regular expression is a literal that is enclosed by the
delimiters ``\lstinline!m{!'' and ``\lstinline!}!''. Regular
expressions are interpreted by the \textit{Perl Compatible
Regular Expression} library (PCRE).\footnote{See
\url{http://www.pcre.org} and
\url{http://www.pcre.org/current/doc/html/pcre2syntax.html}. The
options \textit{PCRE2\_ALT\_BSUX}, \textit{PCRE2\_UCP},
and \textit{PCRE2\_UTF} are set.} Regular expressions
must be encoded in UTF-8\index{UTF-8} and expect the text to be in UTF-8.

Example: \lstinline!m{[a-zA-Z_][a-zA-Z_0-9]*}!

\subsection{Program text literals}
\index{literal!program text}\index{program text literal}\label{textlit}

A program text literal begins with ``\lstinline!q{!'' and is ended
by a balanced ``\lstinline!}!''.\footnote{This type of literal is
perhaps familiar to those knowing Perl. However, its semantics follows
that of the ``\lstinline!qq\{...\}!'' construct in Perl which allows
interpolation.} These literals are designed to support the generation of
program text within the print rules (see \ref{print}).\index{print rule}
They may include embedded references to variables or embedded
expressions that are interpolated when the literal is evaluated. As in
string literals (see \ref{stringlit}), the backslash ``\lstinline!\!''
may be used to escape following characters. Braces may be used without
preceding backslashes within a program text literal as long as they are
balanced.

Example: \lstinline!q{{}}! is equivalent to \lstinline!q{\{\}}!.

Leading and trailing white spaces are removed unless escape sequences
are used to protect them.

\subsubsection{Embedded variable references}

An embedded variable reference begins with the character
``\lstinline!$!''%stopzone
and is followed by an identifier. Spaces between
``\lstinline!$!''%stopzone
and the identifier following it are not permitted.

\subsubsection{Embedded expressions}

An embedded expression begins with the character sequence
``\lstinline!${!'', %stopzone%stopzone
is followed by an arbitrary expression, and closed by
``\lstinline!}!''.

\subsubsection{Interpretation of multi-line literals}

Program text literals may extend over multiple lines. In this
case, the indentations of the individual lines are interpreted
relatively to each other.

Example: Following program text literal generates an
if-else-statement construct that references the variables
\ident{condition}, \ident{then\_statement}, and \ident{else\_statement}.
The white space between the opening ``\lstinline!{!''
and ``if'' is ignored. Similarly, the trailing whitespace between
\lstinline!$else_statement! %stopzone
and the closing
``\lstinline!}!''
is skipped.
However, the relative indentations are noted, i.e.
``\lstinline!$then_statement!'' %stopzone
is to be indented by three additional spaces.

\begin{lstlisting}
   q{
      if ($condition)
         $then_statement
      else
         $else_statement
   }
\end{lstlisting}

\endinput
