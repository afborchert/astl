\chapter{Statements}\index{statement}

\section{Blocks}

A blocks groups a sequence of instructions and opens a nested scope,
i.e. variables declared within a block are not visible outside the
block. Whenever a block is executed, a new instance of it is created
with its own set of local variables (see \ref{var}). Blocks and
their variables are implicitly refered to through closures\index{closure}
(see \ref{function}).

\begin{grammar}
   \nonterminal{block}
      \produces \lextoken{\{} \lextoken{\}} \\
      \produces \lextoken{\{} \nonterminal{statements} \lextoken{\}} \\
   \nonterminal{statements}
      \produces \nonterminal{statement} \\
      \produces \nonterminal{statements} \nonterminal{statement} \\
   \nonterminal{statement}
      \produces \nonterminal{expression} \lextoken{;} \\
      \produces \nonterminal{delete-statement} \lextoken{;} \\
      \produces \nonterminal{if-statement} \\
      \produces \nonterminal{while-statement} \\
      \produces \nonterminal{foreach-statement} \\
      \produces \nonterminal{return-statement} \lextoken{;} \\
      \produces \nonterminal{var-statement} \lextoken{;} \\
\end{grammar}

\section{Deletion statement}
\index{statement!deletion}

A deletion statement allows to delete a key and its entry
in a dictionary. If the given key does not exist, the operation
has no effect.

\begin{grammar}
   \nonterminal{delete-statement}
      \produces \lexkeyword{delete} \nonterminal{designator}
\end{grammar}

\section{Conditional statement}
\index{statement!conditional}\index{conditional statement}

An \keyword{if}-statement allows to execute code conditionally. It
consists of a sequence of conditions and associated blocks with an
optional \keyword{else}-part at the end. The conditions are evaluated
and converted to boolean type (see \ref{boolconv}) in the given order
until one of them evaluates to \ident{true}. Then the associated
block is executed. Otherwise, if none of the conditions evaluates to
\ident{true}, the \keyword{else}-part is executed, if present.

\begin{grammar}
   \nonterminal{if-statement}
      \produces \lexkeyword{if}
	 \lextoken{(} \nonterminal{expression} \lextoken{)}
	 \nonterminal{block} \\
      \produces \lexkeyword{if}
	 \lextoken{(} \nonterminal{expression} \lextoken{)}
	 \nonterminal{block} \lexkeyword{else} \nonterminal{block} \\
      \produces \lexkeyword{if}
	 \lextoken{(} \nonterminal{expression} \lextoken{)}
	 \nonterminal{block} \nonterminal{elsif-chain} \\
      \produces \lexkeyword{if}
	 \lextoken{(} \nonterminal{expression} \lextoken{)}
	 \nonterminal{block} \nonterminal{elsif-chain}
	 \lexkeyword{else} \nonterminal{block} \\
   \nonterminal{elsif-chain}
      \produces \nonterminal{elsif-statement} \\
      \produces \nonterminal{elsif-chain} \nonterminal{elsif-statement} \\
   \nonterminal{elsif-statement}
      \produces \lexkeyword{elsif}
	 \lextoken{(} \nonterminal{expression} \lextoken{)}
	 \nonterminal{block}
\end{grammar}

\section{Loop statements}

\begin{grammar}
   \nonterminal{while-statement}
      \produces \lexkeyword{while}
	 \lextoken{(} \nonterminal{expression} \lextoken{)}
	 \nonterminal{block} \\
   \nonterminal{foreach-statement}
      \produces \lexkeyword{foreach} \nonterminal{identifier}
	 \lexkeyword{in}
	 \lextoken{(} \nonterminal{expression} \lextoken{)}
	 \nonterminal{block} \\
      \produces \lexkeyword{foreach} \lextoken{(}
	 \nonterminal{identifier} \lextoken{,}
	 \nonterminal{identifier} \lextoken{)}
	 \lexkeyword{in}
	 \lextoken{(} \nonterminal{expression} \lextoken{)}
	 \nonterminal{block} \\
\end{grammar}

\noindent
A \keyword{while} statement evaluates first the condition,
converts it to boolean (see \ref{boolconv}). If the condition
is \ident{true} the associated block is executed. This process
repeats until the condition evaluates to \ident{false}.

The first variant of the \keyword{foreach} statement converts its
expression into a list (see \ref{listconv}) and binds the given identifier
to each list member in turn and executes the associated block for it.

The second variant with two variable names expects a dictionary
(see \ref{dictionary}) and binds the given variables to each
key and its associated value in the dictionary and executes the
associated block for it.

\section{Return statement}\label{return}

A return statement is valid within functions only (see \ref{function}
and \ref{functiondef}). In its first variant, \keyword{null} is
returned (see \ref{null}). The second variant evaluates the
given expression and returns its result. In each case, the
execution of the current function is terminated and the execution
continues after the function call.

\begin{grammar}
   \nonterminal{return-statement}
      \produces \lexkeyword{return} \\
      \produces \lexkeyword{return} \nonterminal{expression} \\
\end{grammar}

\section{Variable declarations}\label{var}

Variable declarations bind the given variable name to the value of the
given expression or, if no expression has been given, to \keyword{null}
in the current instance of the local block. Multiple declarations of the
same variable name within the same block are not permitted. Variables
must not be used before being declared and references to equally-named
variables of outer blocks in the same block are prohibited.

\begin{grammar}
   \nonterminal{var-statement}
      \produces \lexkeyword{var} \nonterminal{identifier} \\
      \produces \lexkeyword{var} \nonterminal{identifier}
	 \lextoken{=} \nonterminal{expression} \\
\end{grammar}

\noindent
The binding of a variable can be changed through one of the
assignment operators in a expression statement (see \ref{assignments}).

\endinput
