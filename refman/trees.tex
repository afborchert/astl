\chapter{Tree expressions}\label{treeexpr}\index{tree expression}

Tree expressions are used in attribution rules (see \ref{attr},
transformation rules (see \ref{trans}), and print rules (see
\ref{print}). A tree expression matches or fails for a particular node
in an abstract syntax tree. Whenever a tree expression matches and the
associated rule gets executed, the bindings of the tree expression are
visible in the lexical scope of the rule.

\section{Simple tree expressions}

A simple tree expression consists of a operator set\index{operator
set} (see \ref{opset}) and an arbitrary number of subnodes, specified
by nested tree expressions, variable names, variable-length lists
using the ``...'' operator, or by using the wildcard operator
``*''. Tree expressions containing the operator ``...'' or the wildcard
operator\index{wildcard operator} ``*'' are of variable arity.  A tree
expression matches if and only if the corresponding abstract syntax tree
is an operator node with an operator that is included in the operator
set and where the number of subnodes matches or the arity is variable and
where each of the subtree expressions matches the corresponding subtree.

Regular expressions\index{regular expression} match token nodes only
where the token literal matches the regular expression. String literals
match token only where the token literal is equal to the string literal.

Whenever a named variable is used within \nonterminal{subnode}, it
is, if defined, compared against the corresponding subtree, or, if
undefined, bound to the corresponding subtree. In the former case the
variable can be either an abstract syntax tree or an arbitrary value
that can be converted into a string. In case of an abstract syntax tree,
a comparison covering operators, operands and tokens but not attributes
is done recursively. In case of a string, the match succeeds if and only
if the corresponding subtree represents a token and if the texts are
identical.

Within a list of subnodes, one (and at maximum only one) named variable
may be followed by a ``...'' operator. In this case the variable is
either bound to a list consisting of the remaining subnodes or the list
of remaining subnodes is matched against the already existing list by
comparing the list length and by matching each list element against the
corresponding subnode. This list can be empty if there are no additional
subnodes.

\begin{grammar}
   \nonterminal{tree-expression}
      \produces \lextoken{(} \nonterminal{operator-expression} \lextoken{)} \\
      \produces \lextoken{(} \nonterminal{operator-expression}
	 \nonterminal{subnodes} \lextoken{)} \\
      \produces \lextoken{(} \nonterminal{operator-expression}
	 \lextoken{*} \lextoken{)} \\
   \nonterminal{subnodes}
      \produces \nonterminal{subnode} \\
      \produces \nonterminal{subnodes} \nonterminal{subnode} \\
   \nonterminal{subnode}
      \produces \nonterminal{named-tree-expression} \\
      \produces \nonterminal{regular-expression} \\
      \produces \nonterminal{string-literal} \\
      \produces \nonterminal{identifier} \\
      \produces \nonterminal{identifier} \lextoken{...} \\
   \nonterminal{regular-expression}
      \produces \nonterminal{regexp-literal} \\
      \produces \nonterminal{regexp-literal} \lexkeyword{as}
         \nonterminal{identifier} \\
\end{grammar}

\section{Operator sets}\index{operator set}\label{opexpr}

Operator sets include at least one operator and are defined
through operator expressions:

\begin{grammar}
   \nonterminal{operator-expression}
      \produces \nonterminal{string-literal} \\
      \produces \nonterminal{identifier} \\
      \produces \lextoken{\leftbracketSY}
	 \nonterminal{operators} \lextoken{\rightbracketSY} \\
   \nonterminal{operators}
      \produces \nonterminal{operator-term} \\
      \produces \nonterminal{operators} \nonterminal{operator-term} \\
   \nonterminal{operator-term}
      \produces \nonterminal{string-literal} \\
      \produces \nonterminal{identifier} \\
\end{grammar}

\noindent
Named operator sets can be defined through the \keyword{opset} clause
(see \ref{opset}) which must be declared before it can be used.

\section{Named tree expressions}\index{named tree expression}\index{tree expression!named}

Named tree expressions allow to bind a named variable to a
matched subnode within the tree expression.

\begin{grammar}
   \nonterminal{named-tree-expression}
      \produces \nonterminal{tree-expression} \\
      \produces \nonterminal{tree-expression} \lexkeyword{as}
	 \nonterminal{identifier} \\
\end{grammar}

\section{Contextual tree expressions}\index{tree expression!contextual}\index{contextual tree expression}

Contextual tree expressions allow to specify patterns which must
match for one of the parent nodes. If multiple parent nodes match,
the innermost is taken for the bindings. If multiple contexts
are given in an ``\keyword{and} \keyword{in}''-chain, the
next context is always taken relative to the previous matched context node.
Negated context expressions succeed if no matching parent node is
found. No bindings result from negated context expressions and
every negated context expression resets the context to the original node.

\begin{grammar}
   \nonterminal{contextual-tree-expression}
      \produces \nonterminal{named-tree-expression} \\
      \produces \nonterminal{named-tree-expression}
	 \lexkeyword{in} \nonterminal{context-expression} \\
      \produces \nonterminal{named-tree-expression}
	 \lextoken{!} \lexkeyword{in} \nonterminal{context-expression} \\
   \nonterminal{context-expression}
      \produces \nonterminal{context-match} \\
      \produces \nonterminal{context-match}
	 \lexkeyword{and} \lexkeyword{in} \nonterminal{context-expression} \\
      \produces \nonterminal{context-match}
	 \lexkeyword{and} \lextoken{!} \lexkeyword{in}
	 \nonterminal{context-expression} \\
   \nonterminal{context-match}
      \produces \nonterminal{named-tree-expression} \\
\end{grammar}

\noindent
If a variable named \ident{here} is used within a context expression,
it matches only if the original node or one of its parent nodes is equal
to the node the variable \ident{here} is bound to. If multiple such
special variables are needed in multiple consecutive context expressions,
\ident{here1}, \ident{here2} etc. may be used in this order from left
to right.

Example: Following tree expression matches all identifiers that are
used on the right-hand-side of an assignment expression in C:

\begin{lstlisting}
   ("identifier" id) in (assignment_operators lhs here) as assignment -> {
      println(id, " used on rhs of ", gentext(assignment));
   };
\end{lstlisting}

\noindent
Following example matches all cases where an identifier is within
the same expression once on the left-hand-side of an expression and
then on the right-hand-side:

\begin{lstlisting}
   ("identifier" id)
         in (assignment_operators here1 rhs) as inner_assignment and
         in (assignment_operators lhs here2) as outer_assignment -> {
      println(id, " is assigned to and used in ", gentext(outer_assignment));
   };
\end{lstlisting}

\section{Conditional tree expressions}\index{tree expression!conditional}\index{conditional tree expression}

Conditional tree expressions allow to augment a tree expression with
an expression which may use any of the variables bound in the tree
expression. The conditional expression is evaluated only if the
contextual tree expression matches and before any of the associated
rules are executed.

\begin{grammar}
   \nonterminal{conditional-tree-expression}
      \produces \nonterminal{contextual-tree-expression} \\
      \produces \nonterminal{contextual-tree-expression}
	 \lexkeyword{where} \nonterminal{expression}
\end{grammar}

\endinput
