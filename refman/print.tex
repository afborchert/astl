\chapter{Print rules}\label{print}\index{print rules}

Print rules define how abstract syntax trees are converted into
strings.

\section{Regular print rules}\index{print rules!regular}\label{regprint}

Regular print rules do not belong to a named set of print rules.
They are implicitly used by the \ident{gentext} function
(see \ref{predefined}) and for generating the results of the regular
transformation rules (see \ref{regtrans}).

Print rules consist of
individual rules giving a tree expression (see \ref{treeexpr}) and a
program text literal (see below). The generation of texts starts with
the top-level node passed to \ident{gentext} or at the root node of
the transformed syntax tree. This node is matched
against the available rules. If none matches, a run-time error is
raised, specifying the operator and its arity. If multiple rules
match, the last is taken. The program text literal of the matching
rule specifies the to be generated text. This literal may include
placeholders which are interpolated. If one of these placeholders refers
to a bound subnode, the print rules are executed recursively to generate
the to be interpolated text for the subnode.

\begin{grammar}
   \nonterminal{print-rules}
      \produces \lexkeyword{print} \lexkeyword{rules}
	 \lextoken{\{} \nonterminal{sequence-of-print-rules}
	 \lextoken{\}} \\
   \nonterminal{sequence-of-print-rules}
      \produces \nonterminal{print-rule} \\
      \produces \nonterminal{sequence-of-print-rules}
	 \nonterminal{print-rule} \\
   \nonterminal{print-rule}
      \produces \nonterminal{conditional-tree-expression}
	 \lextoken{->} \nonterminal{print-expression}
\end{grammar}

\noindent
A print expression is a program text literal enclosed in ``q\{'' and
``\}'' with to be interpolated placeholders that begin with the character
``\$''.  If a placeholder references a variable bound to a subnode, it
is expanded by the recursively generated text of that subnode.  In case
of expressions, the result is converted into a string and inserted at
the corresponding position.

In case of multiline program text literals, the leading and trailing
white space is removed and any indentations are taken relatively to
each other.

If a placeholder referencing a variable bound to a list of 
subnodes is followed by the ``\$...'' placeholder, the
sequence is replaced by the empty text if the list is empty,
by the generated text for the first subnode of the list, if
the list has just one subnode, and otherwise expanded
by the generated texts for all subnodes with the text between
the list variable and the ``\$...'' operator inserted between
each of the generated text sequences for the individual subnodes.

\begin{grammar}
   \nonterminal{program-text-literal-placeholder}
      \produces \lextoken{\$} \nonterminal{identifier} \\
      \produces \lextoken{\$}
	 \lextoken{\{} \nonterminal{expression} \lextoken{\}} \\
      \produces \lextoken{\$...}
\end{grammar}

\noindent
Examples: The following print rule generates the text for an
if-statement in C:

\begin{lstlisting}
   ("if" expression then_statement else_statement) -> q{
      if ($expression)
	 $then_statement
      else
	 $else_statement
   }
\end{lstlisting}
%stopzone

\noindent
This print rules supports compound statements with an arbitrary
number of subnodes, each of them representing a statement within
the compound statement:

\begin{lstlisting}
   ("compound_statement" stmt...) -> q{
      {
	 $stmt
	 $...
      }
   }
\end{lstlisting}

\section{Named sets of print rules}\index{print rules!named}\label{named-print}

Named sets of print rules are not implicitly executed. Instead the
given identifier is bound to a function that

\begin{itemize}
   \item expects an abstract syntax tree as parameter or,
      if no parameter is given, uses \ident{root},
   \item recursively applies the print rules as in the
      case of regular print rules, and
   \item returns the generated text as \ident{gentext} for the
      regular print rules.
\end{itemize}

\noindent
A run-time error is raised whenever during the traverse a node
is found without a matching print rule within the set.

\begin{grammar}
   \nonterminal{named-print-rules}
      \produces \lexkeyword{print} \lexkeyword{rules} \nonterminal{identifier}
	 \lextoken{\{} \nonterminal{sequence-of-print-rules}
	 \lextoken{\}} \\
\end{grammar}

\endinput
