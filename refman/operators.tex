\chapter{Operator rules}\label{oprules}

Operator rules, if present, are processed before a program text is
generated using the print rules. Whenever an associativity is expressed
by the abstract syntax tree which would get lost by a program generation
without parenthesizing everything, an operator node is inserted with
``LPAREN'' as operator to override the precedence and associativity of
the language. The operator used for parenthesizing can be changed by
definining an operator set named \ident{parentheses} which consists of
one operator only. This means that, whenever operator rules are employed,
an additional rule is required in the print rules that generates the
required parenthesis, e.g. by

\begin{lstlisting}
   ("LPAREN" expr) -> q{($expr)};
\end{lstlisting}%stopzone

\noindent
or by defining an operator set first (see \ref{opset}):

\begin{lstlisting}
   opset parentheses = ["()"];

   // ...

   print rules {
      ("()" expr) -> q{($expr)};
   }
\end{lstlisting}%stopzone

\begin{grammar}
   \nonterminal{operator-rules}
      \produces \lexkeyword{operators} \lextoken{\{}
	 \nonterminal{operator-lists} \lextoken{\}} \\
   \nonterminal{operator-lists}
      \produces \nonterminal{operator-list} \lextoken{;} \\
      \produces \nonterminal{operator-lists}
	 \nonterminal{operator-list} \lextoken{;} \\
   \nonterminal{operator-list}
      \produces \lexkeyword{left} \nonterminal{operators} \\
      \produces \lexkeyword{right} \nonterminal{operators} \\
      \produces \lexkeyword{nonassoc} \nonterminal{operators} \\
   \nonterminal{operators}
      \produces \nonterminal{operator-term} \\
      \produces \nonterminal{operators} \nonterminal{operator-term} \\
   \nonterminal{operator-term}
      \produces \nonterminal{string-literal} \\
      \produces \nonterminal{identifier} \\
\end{grammar}

\noindent
Operators are to be grouped and sorted by precedence in ascending order.
For each group of operators, the associativity has be to specified
by one of the keywords \keyword{left}, \keyword{right}, or \keyword{nonassoc}.
The keyword is followed by a list of strings representing the
corresponding operators of the abstract syntax tree.

Named operator sets can be defined through the \keyword{opset} clause
(see \ref{opset}) which must be declared before it can be used.

The following example shows the operator rules for the C
programming language:

\begin{lstlisting}
opset assignment_operators = [
   "=" "+=" "-=" "*=" "/=" "%=" "<<=" ">>=" "&=" "^=" "|="
];

operators {
   left ",";
   right assignment_operators;
   right "conditional_expression";
   left "||";
   left "&&";
   left "|";
   left "^";
   left "&";
   left "==" "!=";
   left "<" ">" "<=" ">=";
   left "<<" ">>";
   left "+" "-";
   left "*" "/" "%";

   right "cast_expression";
   right "pointer_dereference" "address_of"
	 "unary+" "unary-" "!" "~" "sizeof"
	 "prefix++" "prefix--";
   left "{}";
   left "postfix++" "postfix--" "function_call"
	 "->" "." "[]";
   nonassoc "label_as_value";
};
\end{lstlisting}

\endinput
