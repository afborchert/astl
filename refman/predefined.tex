\chapter{Predefined bindings}\label{predefined}

Some bindings are provided at a global scope. None of these values
may be redefined at the global level nor modified (i.e. by referencing
them on the left side of an assignment) but it is possible
to hide them by local declarations.

\noindent
\setlength\LTleft{0pt}
\setlength\LTright{0pt}
\begin{longtable}{>{\raggedright\hspace{0pt}}l l p{3.5in}}
   \hline
   name & type & description \\
   \hline
   \endhead
   \hline \multicolumn{3}{r}{\emph{Continued on the next page}}
   \endfoot
   \hline
   \endlastfoot
   \ident{assert} & function &
      aborts the execution if its operand is false \\
   \ident{cfg\_connect} & function &
      creates a directed edge between the two given control flow
      nodes; a label can be optionally specified through a third
      parameter \\
   \ident{cfg\_node} & function &
      creates a control flow node and
      expects a control flow node type in form of a string,
      or an abstract syntax tree, or
      a node type as first and an abstract syntax tree as
      second parameter \\
   \ident{cfg\_type} & function &
      returns the type of a control flow node \\
   \ident{chr} & function &
      returns a string consisting of one Unicode codepoint with
      the value of the first argument converted to an integer;
      an exception is thrown if the integer value is negative
      or too large \\
   \ident{clone} & function &
      creates a clone of dictionaries and lists
      (one level deep copy, not done recursively);
      in all other cases it is similar to a regular assignment \\
   \ident{clone\_ast} & function &
      creates a clone of an abstract syntax tree
      whereas all attributes are copied as well as in
      regular assignments \\
   \ident{copy} & function &
      accepts a target and a source argument which must be both
      non-null and of the same type; the contents of the source
      is then copied to the target \\
   \ident{defined} & function &
      returns true if the value is non-\keyword{null} \\
   \ident{exit} & function &
      expects an argument that is converted to an integer value
      which is taken modulo 256 and interpreted as exit code;
      this function does not return but terminates execution
      with the given exit code \\
   \ident{false} & boolean &
      boolean value of false \\
   \ident{gentext} & function &
      requires the print rules to be available and converts
      an ast node into a string \\
   \ident{integer} & function &
      converts its argument into an integer value \\
   \ident{isoperator} & function &
      its operand must be a node of an abstract syntax tree;
      true is returned if it is an operator node \\
   \ident{isstring} & function &
      returns true if the value is of type string \\
   \ident{len} & function &
      returns the length of the list, or the number of keys
      in a dictionary, or the number of Unicode codepoints
      within a string (with linear complexity), or the
      number of captured substrings in a match result \\
   \ident{location} & function &
      its operand must be a node of an abstract syntax tree;
      returns a string representing its source location \\
   \ident{make\_node}\label{makenode} & function &
      creates an abstract syntax operator node where the
      first parameter specifies the operator and the
      remaining arguments the subnodes; lists of subnodes
      are supported and get expanded \\
   \ident{make\_token}\label{maketoken} & function &
      returns an abstract syntax tree leaf node consisting
      of a token; the token text is derived from the first
      argument which is converted to a string \\
   \ident{operator} & function &
      its operand must be an operator node of an abstract syntax tree;
      returns a string representing its operator \\
   \ident{ord} & function &
      returns the first Unicode codepoint value of the first
      argument which is converted to a string \\
   \ident{pop} & function &
      removes and returns the first element of a list \\
   \ident{push} & function &
      its first operand is a list which is extended by appending
      all the remaining arguments to it \\
   \ident{println} & function &
      prints all arguments and a line terminator to standard output \\
   \ident{prints} & function &
      prints all arguments to standard output \\
   \ident{root}\index{root} & ast & points to the root node of the
      abstract syntax tree; within named attribution rules
      (see \ref{named-atrules}) and transformation rules
      (see \ref{named-trrules} and \ref{named-inplace-trrules})
      \ident{root} is bound to the abstract syntax tree passed
      to the corresponding
      function \\
   \ident{string} & function &
      converts its argument into a string value \\
   \ident{tokenliteral} & function &
      returns the literal text of a token; this usually
      includes the delimiters, e.g. the string quotes in case of
      a string \\
   \ident{tokentext} & function &
      returns the processed text of a token; this usually
      does not include the delimiters or the escape characters \\
   \ident{true} & boolean &
      boolean value of true \\
   \ident{type} & function &
      returns the type of its argument as a string (see \ref{exam-type}) \\
\end{longtable}

\bigskip
\noindent
Whenever print rules are executed for a transformed program text,
some additional predefined bindings are added:

\noindent
\setlength\LTleft{0pt}
\setlength\LTright{0pt}
\begin{longtable}{>{\raggedright\hspace{0pt}}l l p{3.5in}}
   \hline
   name & type & description \\
   \hline
   \endhead
   \hline \multicolumn{3}{r}{\emph{Continued on the next page}}
   \endfoot
   \hline
   \endlastfoot
   \ident{location} & string & the string representation of
      the location of the transformed program text \\
   \ident{rulename} & string & the name of the applied
      transformation rule, if defined and \keyword{null} otherwise. \\
\end{longtable}

\endinput
